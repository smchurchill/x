\startfirstchapter{Introduction}
\label{chapter:introduction}

When we begin studying manifolds, we find the basic problems of surface theory rife with algorithms and elegant solutions.
Moving up by one dimension, a core of algorithms cover 3--manifold theory.
These algorithms, though sophisticated, are historically unexpected in form and possibly hard to implement, leading some to prefer more conceptually pleasant pseudo-algorithms such as SnapPy \cite{culler2017snappy}.
Many problems in high dimensions (5 and up) become impossible due to the complexities of the word problem in fundamental groups and the difficulties in resolving smoothing of CW-complexes.
By comparison, the landscape of 4--manifold theory is not so easily generalized.
High dimensional complications such as the word problem crop up, but dimension 4 is unique when considering the corresponding relation between CW complexes and smoothing theory.
Moreover, algorithmic recognition of the $n$--sphere is resolved in every dimension but 4, where it is an open problem.
Costantino \& Thurston's results in shadow theory resonate traditional 3--manifold theory techniques, suggesting that some basic cobordism questions may not be computationally difficult in dimension 4.


%For 2-manifolds, "clean answers" and algorithms for basic problems are plentiful.
%Even rather high-order problems (like computing metrics on Teichmuller space) are practical.
%In 3-manifold theory, there is a core of algorithms that cover all of the theory, but they are often quite sophisticated and (historically speaking) unexpected in form.
%Depending on the specific topic, the algorithms might be hard to implement, or people might prefer pseudo-algorithms like SnapPea as they are conceptually more pleasant.
%In high dimensions (5 and up) you hit generalities: many questions are impossible to resolve, these come from complexities of the word problem (fundamental groups), or from the complexities associated to CW-complexes (simple homotopy theory) and smoothing theory.
%For 4-manifold theory the landscape for algorithms is un-even and not well understood.
%Elements of high-dimensions pop up, like the word problem via fundamental group issues, but the corresponding relation to CW-complexes and smoothing theory is completely different.
%That said, algorithmic recognition of the 4-sphere is an open problem, where in all other dimensions its resolved.
%And then there is this result of Costantino-Thurston, which one  could view as saying "elements of 3-manifold theory persist" into the theory, that basic cobordism questions appear to not need the computationally-difficult ends of 4-manifold theory.
%i.e. the resolution of this problem given by Thurston and Costantino looks more like traditional 3-manifold theory techniques.

%The study of manifold boundaries is asymmetric.
%Given an $n$--manifold, it is trivial to find its unique $(n-1)$--manifold boundary.
%Given an $(n-1)$--manifold boundary, there are infinitely many $n$--manifolds that possess it but no obvious way to acquire one.
%We focus on the low--dimensional case of $n=4$, where it is known that all closed, orientable 3--manifolds bound 4--manifolds (See cite{wbr61, Rourke85, Thom} for a variety of approaches on this subject).
%The main goal of this document is to provide an algorithm that constructs 4--manifolds with a given boundary.
%
%This century has seen the rise of computational means of constructing and studying low--dimensional manifolds using software such as Regina cite{regina} and SnapPy cite{culler2017snappy}.
%The algorithm provided in this document is intended for eventual inclusion in Regina, where it will serve as a new tool for constructing 4--manifold censuses and studying 3--manifold invariants.

Our assemblage of a constructive proof that 3--manifolds bound 4--manifolds was inspired by the discussion in Section 2.2 of Costantino \& Thurston's ``3--manifolds efficiently bound 4--manifolds'' \cite{CostThur08}.
An appropriately well-behaved smooth map from a 3--manifold $M$ to $\RR$ induces a stratification of $M$ into a union of handlebodies.
This stratification serves as a set of surgery instructions for turning $M$ into $S^3$.
It can also be interpreted as a set of 4--dimensional handle attachment sites, and attaching these handles to one boundary component of $M\times\Ilit$ produces a 4--manifold whose boundary is precisely $M$.

Our algorithm is the adaptation of this smooth proof to the setting of triangulated manifolds.
We define a map from the input 3--manifold triangulation to $\RR$, subdivide the manifold into handlebodies, then attach handles to one boundary component the manifolds' 4--thickening until we obtain the desired outcome.

\section{Expected Background}

This document is aimed at a reader with some understanding of the tools of low--dimensional manifold theory.
We do not present the basics of manifold theory, calculus on manifolds, or triangulations of manifolds.

For the fundamentals of manifolds and calculus on manifolds see Lee's ``Introduction to Smooth Manifolds'' \cite{Lee00}.
For more on 3--manifolds and an introduction to triangulations, see Thurston's ``Three-dimensional geometry and topology'' \cite{thurston1979geometry}.

\section{Agenda}

What follows is a brief summary of the contents of each chapter in this document.
Each chapter has a purpose, and that purpose is also stated.

{\renewcommand\labelitemi{}
	\begin{itemize}
		\item \textbf{Chapter 1} lays out what the central thesis problem is and why it is interesting.  It also sets expectations for the rest of the document in terms of what to expect and what not to expect.
		\item \textbf{Chapter 2} presents some specialized tools of low--dimensional topology that are utilized throughout the document.  It serves as a reference or refresher depending on the reader's familiarity with the subject.
		\item \textbf{Chapter 3} presents the constructive proof in the smooth case.  It fills a void in the current literature and anchors the abstraction of Chapter \ref{chapter:triangulation} to something accessible and visual.
		\item \textbf{Chapter 4} provides the algorithm.  It is a precursor to an implementation of the algorithm in a low--dimensional topology software package.
		\item \textbf{Chapter 5} restates our results and delves into implications on future work.  It, along with Chapter \ref{chapter:introduction} and the Abstract, provide an overview of the entirety of the document.
	\end{itemize}
}