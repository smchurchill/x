\section{Stratification}
\label{section:problem-stratification}

We define stratification using the definition from \cite{wein94}.

\begin{defn}[Stratification]
	$X$ is a \emph{filtered space} on a finite partially ordered indexing set $S$ if 
	\begin{enumerate}
		\item there is a closed subset $X_s$ for each $s\in S$,
		\item $s\leq s'$ implies that $X_s\subset X_{s'}$, and
		\item the inclusions $X_s \into X_{s'}$ satisfy the homotopy lifting property.
	\end{enumerate}
	The $X_s$ are the \emph{closed strata} of $X$, and the differences
	$$X^s = X_s\setminus \bigcup_{r < s} X_r$$
	are \emph{pure strata}.
	
	
	A \emph{filtered map} between spaces filtered over the same indexing set is a continuous function $f:X\to Y$ such that $f(X_s)\subset Y_s$, and such a map is \emph{stratified} if $f(X^s) \subset Y^s$.
	This leads to definitions of stratified homotopy, therefore stratified homotopy equivalence.
\end{defn}

Immediate examples of stratified manifolds are manifolds with boundary and manifolds with corners.
Many handles (e.g.\ $\Ilit\times\Ilit$) are manifolds with corners, and the result of a smooth handle attachment is a manifold with corners at $\varphi(\pd D^\lambda \times \pd D^\mu)$.
Hence both are stratified manifolds.

A \emph{stratified handle attachment} is a handle attachment where the handle, the manifold to which we attach the handle, and the attaching map are each stratified.
The main distinctions between stratified handle attachment and handle attachment are:
\begin{enumerate}
	\item the handle is necessarily stratified, though the stratification is not necessarily induced by the corners that occur in the standard formation of a handle as the Cartesian product of a pair of disks,
	
	\item the manifold to which we attach the handle is necessarily stratified, and
	
	\item the attaching map ensures that there is a coherent identification between the strata of the handle and the strata of the manifold (i.e.\ the stratification of the resulting attachment space is well-defined).
\end{enumerate}
