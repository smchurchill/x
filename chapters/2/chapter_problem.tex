\startchapter{How do you obtain a 4--manifold with a specific boundary?}
\label{chapter:problem}

Given a smooth, closed 3--manifold $M$, there are infinitely many 4--manifolds with boundary $M$.
We do not ensure that the constructed 4--manifold has any properties other than a specified boundary, so our construction ensures easy verification that the constructed 4--manifold's boundary is exactly $M$.
This is done by setting $W=M\times \Ilit$, so $W$ has boundary 
\[
	\pd W = (M\times\{0\}) \cup (M\times\{1\}) = M_0 \cup M_1.
\]
We then attach handles to the boundary of $W$ away from $M_0$ until only $M_0$ remains.

The concept of a stratified handle attachment needs some explanation.
First, we define attachment of topological spaces, and use that language to define handle attachment.

\begin{defn}[Attachment]
	Let $X$ and $Y$ be topological spaces, $A\subset X$ a subspace, and $f:A\to Y$ a continuous map.
	We define a relation $\sim$ by putting $f(x)\sim x$ for every $x$ in $A$.
	Denote the quotient space $X\sqcup Y/\sim$ by $X\cup_f Y$.
	We call the map $f$ the \emph{attaching map}.  
	We say that $X$ is \emph{attached} or \emph{glued} to $Y$ over $A$.
	A space obtained through attachment is called an \emph{adjunction space} or \emph{attachment space}.
	
	Alternatively, we let $A$ be a topological space and let $i_X:A\to X$, $i_Y:A\to Y$ be inclusions.
	Here, the adjunction is formed by taking $i_X(a)\sim i_Y(a)$ for every $a\in A$ and we denote the adjunction space by $X\cup_A Y$.
\end{defn}

\begin{defn}[Handle]
	\label{def:handle}
	Take $n=\lambda+\mu$ and $M$ a smooth $n$--manifold with nonempty boundary $\pd M$.
	Let $D^\lambda$ be the closed $\lambda$--disk and put $H^\lambda = D^\lambda\times D^\mu$.
	Let $\varphi:\pd D^\lambda\times D^\mu\to\pd M$ be an embedding and an attaching map between $M$ and $H^\lambda$.
	The attached space $H^\lambda$ is an \emph{$n$--dimensional $\lambda$--handle}, and $M\cup_\varphi H^\lambda$ is the result of an $n$--dimensional \emph{$\lambda$--handle attachment}.
\end{defn}

Handle attachment is defined for smooth manifolds, but the resulting attachment space is not a smooth manifold.
Rather, the result is a stratified manifold.
We use the definition from \cite{wein94}.

\begin{defn}[Stratification]
	$X$ is a \emph{filtered space} on a finite partially ordered indexing set $S$ if 
	\begin{enumerate}
		\item there is a closed subset $X_s$ for each $s\in S$,
		\item $s\leq s'$ implies that $X_s\subset X_{s'}$, and
		\item the inclusions $X_s \into X_{s'}$ satisfy the homotopy lifting property.
	\end{enumerate}
	The $X_s$ are the \emph{closed strata} of $X$, and the differences
	$$X_s\setminus \bigcup_{r < s} X_r$$
	are \emph{pure strata}.
	The pure strata are denoted $X^s$.
	The singular declension of the word strata is \emph{stratum}.
	
	A \emph{filtered map} between spaces filtered over the same indexing set is a continuous function $f:X\to Y$ such that $f(X_s)\subset Y_s$, and such a map is \emph{stratified} if $f(X^s) \subset Y^s$.
	This leads to definitions of stratified homotopy, therefore stratified homotopy equivalence.
\end{defn}

Immediate examples of stratified manifolds are manifolds with boundary and manifolds with corners.
Many handles (e.g.\ $D^1\times D^1$) are manifolds with corners, and the result of a smooth handle attachment is a manifold with corners at $\varphi(\pd D^\lambda \times \pd D^\mu)$.
Hence both are stratified manifolds.

A \emph{stratified handle attachment} is a handle attachment where the handle, the manifold to which we attach the handle, and the attaching map are each stratified.
The main distinctions between stratified handle attachment and handle attachment are:
\begin{enumerate}
	\item the strata of the handle include, but are not exclusive to, the naturally occurring corners from the formation of the handle as the Cartesian product of a pair of disks,
	
	\item the manifold to which we attach the handle is necessarily stratified, and
	
	\item the attaching map ensures that there is a coherent identification between the strata of the handle and the strata of the manifold (i.e.\ the stratification of the resulting attachment space is well-defined).
\end{enumerate}



