\section{Stratifying $M$}

Decomposing $\RR$ via the singular values of $f$ also induces a stratification of $M$
by considering the fibers of $f$ above the of the decomposing regions.
The interiors of face regions have preimage through $f$ a disjoint collection of face blocks, the interiors of edge regions have preimage of edge blocks, and of vertex regions, vertex blocks.

To understand the structure of face, edge, and vertex blocks we investigate the preimages of regular and singular values of $f$ in the plane.
\begin{defn}
	Because $M$ is closed, $f$ is proper.
	Thus, for any point $q$ in $f(M)$, a fiber of $f$ above $q$ (i.e.\ a connected component of $f\inv(q)$) is either a closed 1--manifold (i.e.\ $S^1$) or contains a critical point of $f$.
	
	We define a \emph{singular fiber} to be a fiber that contains a critical point of $f$, and a \emph{regular fiber} to be a fiber consisting entirely of regular points.	
\end{defn}

The subsets used to stratify $M$ are the fibers of $f$ above the corners of the octagonal vertex regions, the edges of the vertex and edge regions, and the regions themselves.
Because the corners of the vertex regions are regular values, their fibers are regular, hence a finite collection of disjointly embedded circles in $M$.
We take these circles as the first collection of subsets that filter $M$, and assign to them the indices $(1,i)$ for $i = 1\dots N_1$, where $N_1$ is the number of circles.
These circles are disjoint, so $(1,i)\nleq (1,j)$ for any $i,j$.

The arcs of the decomposition connect the vertices and either consist entirely of regular values or contain exactly one singular value.
When an arc contains exactly one singular value, exactly one fiber above that value is a singular fiber, with the rest regular fibers.
A decomposing arc is diffeomorphic to the unit interval and $f$ is a smooth submersion between smooth manifolds, so a fiber above a decomposing arc is a surface with boundary the fibers above the arc endpoints.
When the fiber is regular, the surface is diffeomorphic to an annulus $S^1\times\Ilit$.
When the fiber is singular, the surface classification depends on the type of singularity.
The following theorem and illustration (Theorem 3.15 from \cite{Saeki}) shows that the fiber containing the singularity either has the structure of a figure-of-eight or is a single point, hence the singular fiber above the arc is diffeomorphic to either a 2--disc or a pair of pants.

\begin{theorem}[Adapted Theorem 3.15 in Saeki \cite{Saeki}]
	Let $f:M\to N$ be a proper $C^\infty$ stable map of an orientable 3--manifold $M$ into a surface $N$.
	Then, every singular fiber of $f$ is equivalent to the disjoint union of one of the fibers as in Figure \ref{fig:saeki-fibers} and a finite number of copies of a fiber of the trivial unit circle bundle.
	Furthermore, no two fibers in the list are equivalent to each other even after taking the union with regular circle components.		
\end{theorem}

\begin{figure}[h!]
	\caption{
		\textbf{Singular fibers.}
		List of singular fibers of proper $C^\infty$ stable maps of orientable 3--manifolds into surfaces.
		$\kappa$ is the codimension of the singularity in the surface.
	}
	\label{fig:saeki-fibers}	
\end{figure}

The surface fibers above the decomposing arcs are the second collection of subsets that filter $M$, and they are assigned the indices $(2,j)$ for $j=1\dots N_2$, where $N_2$ is the number of surfaces.
The boundary circles of the surfaces are each subsets of the filtration, indexed by the $(1,i)$ indices, so $(1,i)\leq (2,j)$ if and only if $M_{(1,i)}$ is one of the boundary components of $M_{(2,j)}$.
These surfaces intersect one another only when they share a boundary circle, so $(2,j)\nleq (2,k)$ for any $j,k$.

There are three types of region in the decomposition: face, edge, and vertex.
Regardless of the type of region, they are indexed in our filtration similarly to the edges.
A fiber above a region is a 3--manifold with corners formed by the $(1,i)$- and $(2,j)$-level strata, and fibers are disjoint away from their boundaries.
We therefore index fibers above regions with the indices $(3,k)$ for $k=1\dots N_3$ where $N_3$ is the total number of fibers above regions, put $(n,i)\leq (3,k)$ if and only if $M_{(n,i)}$ is contained in the boundary of $M_{(3,k)}$.

Recall that we call a closed 3--dimensional stratum of $M$ a \emph{block}.
We now prove that the stratification conditions impose the desired structure on the blocks of $M$, as discussed in the introduction to this chapter.

\begin{theorem}
	\label{thm:block-structure}
	Let $M$ be a closed, smooth, orientable 3--manifold, let $f:M\to\RR$ be a map satisfying the \emph{stratification conditions} of Section \ref{section:smooth-projection}, suppose $\RR$ has been decomposed as in Section \ref{section:smooth-decompose} and $M$ has been stratified as in this section.
	Then each closed strata $M_{(3,k)}$ is classified as either a \emph{face}, \emph{edge}, or \emph{vertex block} depending on whether it is a fiber above a face, edge, or vertex region respectively, and a block has one of the following structures:
	\begin{enumerate}
		\item[\emph{face block}:]
		Let $B$ be a face block and let $F$ be the face region that $B$ is a fiber over.
		Then $B$ is homeomorphic to $S^1\times F$.
		
		\item[\emph{edge block}:]
		Let $B$ be an edge block and let $E$ be the edge region that $B$ is a fiber over.
		Let $A$ be the annulus $S^1\times\Ilit$ and $P$ the pair-of pants surface (i.e.\ $D^2$ minus a pair of disjoint open balls).
		If $B$ is a regular fiber over $E$ then $B$ is homeomorphic to $S^1\times E$, hence also homeomorphic to $A\times\Ilit$.
		Otherwise, $B$ is a singular fiber over $E$ and contains part of definite or indefinite fold.
		In this case we call $B$ a \emph{definite} or \emph{indefinite edge block}.
		A definite edge block is homeomorphic to $D^2\times\Ilit$ and an indefinite edge block is homeomorphic to $P\times\Ilit$.

		\item[\emph{vertex block}:]
		Let $B$ be a vertex block and let $V$ be the vertex region that $B$ is a fiber over.
		If $B$ is a regular fiber then it is homeomorphic to $S^1\times V$, therefore homeomorphic to a $(3,1)$-handlebody of genus 1.
		Otherwise, we see from Figure \ref{fig:saeki-fibers} that the singular fiber above the codimension 2 singularity contained in $V$ is either connected or disconnected.
		If the singular fiber is disconnected then there are a pair of disjoint vertex blocks that each contain one of the singular fibers, hence part of a definite or indefinite fold.
		We therefore classify these blocks as \emph{definite} or \emph{indefinite vertex blocks}.
		If the singular fiber is connected, then the block containing it is an \emph{interactive vertex block}.
		A definite(resp.\ indefinite) vertex block extends and connects definite(resp.\ indefinite) edge blocks, and is homeomorphic to a $(3,1)$-handlebody of genus 0(resp.\ 2).
		An interactive vertex block is homeomorphic to a $(3,1)$-handlebody of genus 3.
	\end{enumerate}
\end{theorem}

\begin{rmk}
	The structures of the blocks described in Theorem \ref{thm:block-structure} are roughly disc bundles over a representative fiber for the given region.
	For a block that is a regular fiber, the representative is a circle.
	For a definite or indefinite block, the representative is the fiber containing a definite or indefinite fold, and for an interactive block the representative fiber is the singular fiber above the codimension 2 singular value.
\end{rmk}

\begin{proof}[Proof of Theorem \ref{thm:block-structure}]
	We split the proof into three parts.
	The first part proves that if a block $B$ is a regular fiber over the region $R$ then $B$ is homeomorphic to $S^1\times R$.
	In the second part, we prove that definite and indefinite blocks are homeomorphic to $D^2\times\Ilit$ or $P\times\Ilit$ respectively.
	In the final part we discuss interactive vertex blocks, and show that they are homeomorphic to $(3,1)$-handlebodies of genus 3.
	Figures are provided in each part.
	
	\textbf{Part 1:}
	Let $B$ be a block over a region $R$, and suppose $B$ consists entirely of regular fibers over $R$.
	Then $(B, R, f|_B, S^1)$ has the structure of a circle bundle over $R$.
	A fiber bundle over a contractible space is trivial, so $B$ is homeomorphic to $S^1 \times R$.
	Furthermore, this homeomorphism is stratified by ensuring the strata of $B$ are mapped to the strata of $S^1 \times R$, where the stratification of $S^1 \times R$ is defined by the manifold with corners structure induced by the product topology.
	See Figure \ref{fig:regular-blocks}.
	
	\begin{figure}[h!]
		\caption{
			\textbf{Regular blocks.}
			Three types of regular blocks.
			These are found as regular fibers over face, edge, and vertex regions.
		}
		\label{fig:regular-blocks}
	\end{figure}
	
	\textbf{Part 2:}
	Let $B$ be a definite block over a region $R$.
	Then $B$ contains an arc $\gamma$ that is part of a definite fold in $S_1(f)$, and $f(\gamma)$ is an arc of codimension 1 singular values that runs from one edge of $R$ to another (see Figure \ref{fig:definite-blocks}).
	By definition, for every point $p$ of $\gamma$ there exist local coordinates $(x,z_1,z_2)$ at $p$ and $(x,y)$ at $f(p)$ such that 
	\[(x,y) = (u,\pm(z_1^2+z_2^2)).\]
%	First, we need to connect the singular fibers to the type of fold
	Let $\nu(\gamma)$ be a tubular neighbourhood around $\gamma$ inside of $B$ with these coordinates, and let $\bar\nu(\gamma)$ be its closure.
	These coordinates ensure that $f(B\setminus\nu(\gamma))$ falls entirely on one side of $f(\gamma)$ in $R$.
	By Part 1 of this proof, $B\setminus\nu(\gamma)$ is a circle bundle over the image of $B\setminus\nu(\gamma)$ through $f$, while $\bar\nu(\gamma)$ is .
	The local coordinates around $\gamma$ tell us that the circles collapse into single points, yielding the singular fibers
	
	
	\begin{figure}[h!]
		\caption{
			\textbf{Definite blocks.}
			The two types of definite blocks.
			These are found as singular fibers over edge and vertex regions.
		}
		\label{fig:definite-blocks}
	\end{figure}

	\begin{figure}[h!]
	\caption{
		\textbf{Inefinite blocks.}
		The two types of indefinite blocks.
		These are found as singular fibers over edge and vertex regions.
	}
	\label{fig:indefinite-blocks}
	\end{figure}
	
	
	\textbf{Part 3:}
\end{proof}
