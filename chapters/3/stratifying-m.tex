\section{Stratifying $M$}

Decomposing $\RR$ via the singular values of $f$ also induces a stratification of $M$
by considering the fibers of $f$ above the of the decomposing regions.
The interiors of face regions have preimage through $f$ a disjoint collection of face blocks, the interiors of edge regions have preimage of edge blocks, and of vertex regions, vertex blocks.

To understand the structure of face, edge, and vertex blocks we investigate the preimages of regular and singular values of $f$ in the plane.
\begin{defn}
	Because $M$ is closed, $f$ is proper.
	Thus, for any point $q$ in $f(M)$, a fiber of $f$ above $q$ (i.e.\ a connected component of $f\inv(q)$) is either a closed 1--manifold (i.e.\ $S^1$) or contains a critical point of $f$.
	
	We define a \emph{singular fiber} to be a fiber that contains a critical point of $f$, and a \emph{regular fiber} to be a fiber consisting entirely of regular points.	
\end{defn}

The subsets used to stratify $M$ are the fibers of $f$ above the corners of the octagonal vertex regions, the edges of the vertex and edge regions, and the regions themselves.
Because the corners of the vertex regions are regular values, their fibers are regular, hence a finite collection of disjointly embedded circles in $M$.
We take these circles as the first collection of subsets that filter $M$, and assign to them the indices $(1,i)$ for $i = i\dots N_1$, where $N_1$ is the number of circles.
These circles are disjoint, so $(1,i)\nleq (1,j)$ for any $i,j$.

