\chapter{Proof for smooth manifolds}
\label{chapter:smooth}

We prove that every smooth, closed, orientable 3--manifold is the boundary of some 4--manifold.
We do so by explicitly constructing such a 4--manifold from a given 3--manifold.
This construction is mirrored in Chapter \ref{chapter:triangulation} where we prove the same for a given closed, orientable 3--manifold triangulation and provide an algorithm.

Let $M$ be a smooth, closed, orientable 3--manifold and take $W=M\times \Ilit$.
Then $W$ is a 4--manifold with boundary
\[
	\pd W = (M\times\{0\}) \cup (M\times\{1\}) = M_0 \cup M_1.
\]
We construct a manifold with only one boundary component from $W$ by attaching 4--dimensional 2--, 3--, and 4--handles to $W$ over the part of its boundary away from $M_0$.
We start by attaching 2--handles to $W$ over $M_1$ to produce a 4--manifold $W'$ with boundary $M_0 \sqcup M_1'$, where $M_1'=\pd W'\setminus M_0$ and is described via surgery on $M_1$.
We then attach 3--handles to $W'$ over $M_1'$ to produce a 4--manifold $W''$ with boundary $M_0\sqcup M_1''$.
As before, $M_1''=\pd W''\setminus M_1$ and $M_1''$ is described via surgery on $M_1'$.
At this point in the construction, $M_1''$ is the disjoint union of a finite number of copies of $S^3$.
We attach 4--handles to $W''$ over $M_1''$ to produce a 4--manifold whose boundary is exactly $M_0$.

Instructions for handle attachment come from defining a projection $f:M_1\to \RR$ that induces a stratification of $M_1\subset\pd W$.

%
% indexed partially by dimension of the pure strata as a submanifold of $M_1$.
% such that each closed stratum $M_1^3$ is a disjoint collection of open submanifolds of $M_1$.
%
We call a closed 3--dimensional stratum of $M_1$ a \emph{block}, and we impose conditions on $f$ to ensure that every block can be classified up to \emph{stratified homeomorphism}.
We say that $X,Y$ are \emph{stratified homeomorphic} if $X,Y$ are stratified spaces and there exists a homeomorphism $f:X\to Y$ such that $f$ and $f\inv$ are each stratified maps.
The blocks of $M_1$ are described below:

{\renewcommand\labelitemi{}
\begin{itemize}
	\item \textbf{Face block:}
		An attachment neighbourhood for a stratified 2--handle.
		Each face block is stratified-homeomorphic to $S^1\times G_n$, the product of the circle with an $n$-gon for some $n$.
	
	\item \textbf{Edge block:}
		A partial attachment neighbourhood for a stratified 3--handle.
		Each edge block is stratified-homeomorphic to one of $D^2\times\Ilit$, $A\times\Ilit$, or $P\times\Ilit$, where $A$ is the annulus $S^1\times\Ilit$ and $P$ is a pair-of-pants surface.
		Attachment of stratified (4,2)--handles over our face blocks ``fill in'' the annular boundary components of edge blocks, forming full attachment neighbourhoods for stratified 3--handles.
		
	\item \textbf{Vertex block:}
		A partial attachment neighbourhood for a stratified 4--handle.
		Each vertex block is homeomorphic to a (3,1)-handlebody of genus at most 3.
		When stratified (4,2)-- and (4,3)--handles are attached to $W$, the genus of these handlebodies are reduced until the remaining boundary of $W$ consists of $M_0$ union a collection of stratified 3--spheres.  The 3--spheres are then coned away.
\end{itemize}
}
 
The remainder of this chapter is spent ensuring that such a stratification can be achieved for any smooth, orientable, closed 3--manifold, detailing how the stratification is induced, proving that the attachment of stratified (4,2)-- and (4,3)--handles has the previously stated effects, and discussing the resulting 4--manifold.

\section{Projections from 3--manifolds to $\RR$}
\label{section:smooth-projection}

Our stratification of $M$ is induced by a decomposition of the plane, itself induced by the singular values of a smooth map $M\to\RR$.
To prove that a stratification suitable for our construction exists for any smooth orientable 3--manifold, we show first that an inducing decomposition of $\RR$ exists.
To prove that such a decomposition of $\RR$ exists, we present the properties of $f:M\to\RR$ required to induce the decomposition, and argue why a map possessing such properties exists for any smooth, orientable 3--manifold.

Let $f:M\to\RR$ be a smooth map, let $df$ be the differential of $f$, and let $S_r(f)$ be the set of points in $M$ such that $df$ has rank $r$.
Then we require that the following be true of $f$:
\begin{enumerate}
	\item $S_0(f)$ is empty.
	
	\item $S_1(f)$ consists of smooth non-intersecting curves.  We call these the \emph{fold curves} of $f$.
	
	\item The set of points where $f|_{S_1(f)}$ has zero differential is empty.
	
	\item\label{strat-con:fold-type} If $p\in S_1(f)$ then there exist coordinates $(u, z_1, z_2)$ centred at $p$ and $(x,y)$ centred at $f(p)$ such that $f$ takes the form of either
	\begin{enumerate}
		\item\label{property:paraboloid} $(x,y)=(u,\pm(z_1^2+z_2^2))$, or
		\item\label{property:saddle} $(x,y)=(u,\pm(z_1^2-z_2^2))$
	\end{enumerate}
	in a neighbourhood of $p$.
	If $f$ takes the form of \ref{property:paraboloid} then we further classify $p$ as a \emph{definite fold}, and if $f$ takes the form of \ref{property:saddle} then $p$ is an \emph{indefinite fold}.
	
	\item Let $\gamma_i, \gamma_j, \gamma_k \in S_1(f)$ be fold curves. Then each of $f(\gamma_i)$, $f(\gamma_j)$, $f(\gamma_k)$ are submanifolds of $\RR$ such that
	\begin{enumerate}
		\item $f(\gamma_i)$ and $f(\gamma_j)$ intersect transversely,
		
		\item $f(\gamma_i)\cap f(\gamma_j)\cap f(\gamma_k)$ is empty (i.e. there are no triple-intersections), and
		
		\item self-intersections of $f(\gamma_i)$ are transverse
	\end{enumerate}
	
	\item The set of singular values of $f$ in the plane is connected
	
	\item The image of $M$ through $f$ is bounded in the plane.
	
% f is continuous and M is compact so f(M)is compact in R^2 (ie is closed and bounded)
\end{enumerate}

%The first four conditions ensure that the set of singular values in the plane is well-behaved, the fifth allows classification of singular fibers of $f$ using the work of Saeki \cite{Saeki}, and the sixth guarantees that the set of singular values of $f$ partitions the image of $M$ into discs.
We call these the \emph{stratification conditions} on $f$, and we call a map satisfying the stratification conditions a \emph{stratifying map}.
%Each stratification condition simplifies the arguments of this chapter in some way.
%The first three conditions simplify the description of $X_f=f(S(f))$, ensuring that $X_f$ is describable as a collection of smooth closed curves in the plane.
%The fourth condition simplifies the discussion of neighbourhoods of singular fibers of $f$.
%Condition 5 ensures that crossings of $X_f$ occur uniformly, simplifying the plane decomposition of Section \ref{section:smooth-decompose}.
%The final two conditions ensure that $\RR\setminus X_f$ consists of a finite collection of open surfaces $\Sigma_i$, $i=1\dots k$ each homeomorphic to a 2--disk, and along with $\Sigma_\infty$ which is unbounded.
%Furthermore, each $\Sigma_i$ is contained in the image of $f$ whereas $\Sigma_\infty\cap f(M)$ is empty.
The existence of smooth maps satisfying conditions 1--4 is discussed in \cite{Levine65}, and 5--7 are straightforward.
A smooth map satisfying conditions 1--4 can be smoothly perturbed inside of a tubular neighbourhood of $S_1(f)$ to satisfy conditions 5 \& 6.
Finally, condition 7 is always satisfied because the image of a compact set through a continuous map is compact.

% 1-4 from lvine65
% 5-7 argue.

%Before stratifying the plane with a stratifying map, let's discuss some of the implications of the stratification conditions.
Let $f:M\to\RR$ be a stratifying map and let $X_f = f(S(f))$, the set of singular values of $f$.
We first consider $f(M)\setminus X_f$.
Because $X_f$ is a connected collection of arcs in the plane that intersect only transversely, $f(M)\setminus X_f$ is a collection of connected regions in the plane, each of which consists entirely of regular values.
The sixth stratification condition guarantees that each of these regions is simply connected --- the boundary components of these regions are formed by the singular values of $f$ in the plane, and different boundary components are necessarily disjoint.
The sixth stratification condition does not restrict the class of maps we consider, as a map satisfying the first five stratification conditions can be smoothly homotoped to satisfy the sixth.

%Our 4--manifold construction assumes that there is at least one arc intersection in $X_f$.
%Fortunately, there is only one example to consider when this is not the case: when $X_f$ is a simple closed curve in the plane.



\section{Stratifying $\RR$}
\label{section:smooth-decompose}

Let $f:M\to\RR$ be a smooth map possessing the stratification conditions of Section \ref{section:smooth-projection} and let $X_f = f(S(f))$, the set of singular values of $f$.
$X_f$ is a connected collection of arcs in the plane that intersect only transversely.

We fit closed neighbourhoods (\emph{sleeves}) around the singular values of $f$ and classify these sleeves by the maximum codimension (with respect to $\RR$) of singular values they contain.
Because $X_f$ consists of codimension 1 and codimension 2 singular values (i.e.\ arcs and arc-crossings respectively), we stratify $\RR$ into face regions that contain no singular values, edge regions that contain only codimension 1 singular values, and vertex regions, each of which contain exactly 1 codimension 2 singular value.
Figures \ref{fig:vertex-sleeve}-\ref{fig:face-sleeve} are used to illustrate the stratification resulting from sleeve-fitting.

\begin{figure}[h!]
	\centering
	\includegraphics[width=\textwidth]{figures/vertex-sleeve.png}
	\caption{
		\textbf{Forming vertex regions.}
		Octagonal sleeves are fit around codimension 2 singular values to form vertex regions.
	}
	\label{fig:vertex-sleeve}
\end{figure}

We begin by fitting octagonal sleeves around codimension 2 singular values as in Figure \ref{fig:vertex-sleeve}.
%Octagons are used here solely to simplify descriptions further down the line of proof.
Let $x$ be a codimension 2 singular value.
$x$ is the result of an arc crossing, and a small neighbourhood around an arc crossing is divided into four regions of regular values.
The octagon around $x$ is fit so its edges alternate between being fully contained in a region of regular values and orthogonally intersecting one of the arcs of singular values that creates $x$.
See Figure \ref{fig:vertex-sleeve} for a model fitting.

The interiors of the octagons along with the octagonal boundaries form the vertex regions of the stratification of $\RR$.
The octagons are chosen to be small enough that no two vertex regions overlap and such that the octagonal edges that intersect arcs of codimension 1 singular values are all the same length.

\begin{figure}[h!]
	\centering
	\includegraphics[width=\textwidth]{figures/edge-sleeve.png}
	\caption{
		\textbf{Forming edge regions.}
		Vertex region corners are connected to fit sleeves around arcs of codimension 1 singular values to form edge regions.
	}
	\label{fig:edge-sleeve}
\end{figure}

Let $\gamma$ be an arc of codimension 1 singular values with endpoints a pair codimension 2 singular values.
$\gamma$ orthogonally intersects one edge from each of the octagonal vertex regions fit around its endpoints, and we use these edges to form the edge region associated to $\gamma$ by connecting the endpoints of these edges to one another using a pair of arcs parallel to $\gamma$.
See Figure \ref{fig:edge-sleeve} for a model fitting.

The closures of the interiors of the shapes formed by the arcs and octagon edges form the edge regions of the stratification of $\RR$.
The octagonal edge endpoints are also vertices of the octagons, and the formation of edge regions uses every octagonal vertex as the endpoint of exactly one arc.

\begin{figure}[h!]
	\centering
	\includegraphics[width=\textwidth]{figures/face-sleeve.png}
	\caption{
		\textbf{Forming face regions.}
		All remaining regions contain no singular values, and we take these to be the face regions.
	}
	\label{fig:face-sleeve}
\end{figure}

Removing from $f(M)$ all vertex and edge regions, we are left with a collection of simply connected regions in the plane, each of which consists entirely of regular values.
We take the closures of these to be the face regions of the stratification of $\RR$.
The boundary of each face region is an alternating collection of arcs from edge regions and octagonal edges from vertex regions.
See Figure \ref{fig:face-sleeve} for a model fitting.

With all of the regions defined, we can describe precisely how $f$ stratifies $\RR$.
The stratification of $\RR$ is a stratification into subsets $R_{(i,j)}$ where $i,j$ are integers.
Subset indexing is defined so that a subset $R_{(i,j)}$ is an $i$--dimensional submanifold of $\RR$, thus $R_{(i,j)}\nleq R_{(k,l)}$ if $i\nleq k$.
The first collection of subsets used to filter $\RR$ are the corners of the octagonal vertex sleeves.
We assign to these subsets the indices $(0,i)$ for $i=1\dots N_0$, where $N_0$ is the number of corners.
Corners are disjoint, hence $R_{(0,i)}$ is not contained in $R_{(0,j)}$ for any $i, j$, whence $(0,i)\nleq (0,j)$ for any $i,j$.

The boundary arcs connect the $(0,i)$--level strata.
Arcs are indexed by $(1,j)$ for $j=1\dots N_1$, where $N_1$ is the number of arcs.
The boundary points of an arc are corners and are subsets of the filtration indexed by the $(0,i)$ indices, so $(0,i)\leq (1,j)$ if and only if $R_{(0,i)}$ is one of the boundary points of $R_{(1,j)}$.
Arcs intersect only at their boundary points, so $(1,j)\nleq(1,k)$ for any $j,k$. 

The regions themselves are indexed by $(2,k)$ for $k=1\dots N_2$, where $N_2$ is the number of regions.
These indices work similarly to the arc indices.
The boundary of a region consists of corners and arcs, so $(n,i)\leq(2,k)$ if and only if $R_{(n,i)}$ is contained in the boundary of $R_{(2,k)}$.

With $\RR$ stratified, we move onto a stratification of $M$.
This stratification is induced by the preimages of the strata of $\RR$.

%
%Before moving on to the stratification of $M$ induced by our decomposition of $\RR$, we need to iron out the corner cases of when $X_f$ contains no codimension 2 singular values.
%
%
%Because $X_f$ is connected, $X_f=f(S_1(f))$, and $S_1(f)$ is a collection of smooth non-intersecting curves in $M$, we conclude that $S_1(f)$ contains exactly one curve.
%Furthermore, $f(M\setminus S_1(f))$ lies entirely within $X_f$ so $S_1(f)$ is everywhere a definite fold.
%
%...



\section{Stratifying $M$}

Stratifying $\RR$ via the singular values of $f$ also induces a stratification of $M$
by considering the fibers of $f$ above the stratifying regions.
The interiors of face regions have preimage through $f$ a disjoint collection of face blocks, the interiors of edge regions have preimage of edge blocks, and of vertex regions, vertex blocks.

To understand the structure of face, edge, and vertex blocks we investigate the preimages of regular and singular values of $f$ in the plane.
\begin{defn}
	Because $M$ is closed, $f$ is proper.
	Thus, for any point $q$ in $f(M)$, a fiber of $f$ above $q$ (i.e.\ a connected component of $f\inv(q)$) is either a closed 1--manifold (i.e.\ $S^1$) or contains a critical point of $f$.
	
	We define a \emph{singular fiber} to be a fiber that contains a critical point of $f$, and a \emph{regular fiber} to be a fiber consisting entirely of regular points.	
\end{defn}

The subsets used to stratify $M$ are the fibers of $f$ that lie above the individual strata of our $\RR$ stratification.
Because the $(0,\cdot)$--strata are regular values, their fibers are regular, hence a finite collection of disjointly embedded circles in $M$.
We take these circles as the first collection of subsets that filter $M$, and assign to them the indices $(1,i)$ for $i = 1,\dots, N_1$, where $N_1$ is the number of circles.
These circles are disjoint, so $(1,i)\nleq (1,j)$ for any $i,j$.

The $(1,\cdot)$--strata of the $\RR$ stratification connect the $(0,\cdot)$--strata of the $\RR$ stratification and either consist entirely of regular values or contain exactly one singular value.
As pictured in Figure \ref{fig:face-sleeve}, a $(1,\cdot)$--strata contains exactly one singular value precisely when it is the shared boundary of a vertex region and an edge region.
When a $(1,\cdot)$--strata contains exactly one singular value, exactly one fiber above that value is a singular fiber, with the rest regular fibers.
A $(1,\cdot)$--strata is diffeomorphic to the unit interval and $f$ is a smooth submersion between smooth manifolds, so a fiber above a $(1,\cdot)$--strata is a surface whose boundary circles are the fibers above the strata's endpoints.
When the fiber is regular, the surface is diffeomorphic to an annulus $S^1\times\Ilit$.
When the fiber is singular, the surface classification depends on the type of singularity.
Theorem \ref{thm:saeki} and Figure \ref{fig:saeki-fibers} show that the fiber containing the singularity either has the structure of a figure-of-eight (when the singularity is part of an indefinite fold) or is a single point (when the singularity is part of a definite fold), hence the singular fiber above the arc is diffeomorphic to either a 2--disk or a pair-of-pants.

Theorem \ref{thm:saeki} refers to a \emph{stable map}, and we'll denote the set of smooth stable maps $X\to Y$ by $Stab(X,Y)$.
When $X$ is a smooth, closed, orientable 3--manifold and $Y$ is the plane, $Stab(X,Y)$ consists of all maps $X\to Y$ satisfying the first five stratification conditions.
The last stratification condition is trivially satisfied because $X$ is closed, hence the set of stratifying maps $X\to Y$ is the subset of $Stab(X,Y)$ consisting of maps $f$ such that $f(S(f))$ is connected.

\begin{theorem}[Adapted Theorem 3.15 in Saeki \cite{Saeki}]
	\label{thm:saeki}
	Let $f:M\to N$ be a proper $C^\infty$ stable map of an orientable 3--manifold $M$ into a surface $N$.
	Then, every singular fiber of $f$ is equivalent to the disjoint union of:
	\begin{enumerate}
		\item one of the fibers in Figure \ref{fig:saeki-fibers}, and
		\item the disjoint union of a finite number of copies of $S^1$.
	\end{enumerate}
	Furthermore, no two fibers in the list are equivalent to each other even after taking the union with regular circle components.		
\end{theorem}

\begin{figure}[h!]
	\centering
	\includegraphics{figures/saeki-fibers.png}
	\caption{
		\textbf{Connected singular fibers.}
		List of connected singular fibers of proper $C^\infty$ stable maps of orientable 3--manifolds into surfaces.
		$\kappa$ is the codimension of the singularity in the surface.
		The singular fiber above a codimension 2 singular value may be disconnected,
		in which case the fiber is the disjoint union of a pair of singular fibers from $\kappa=1$.
	}
	\label{fig:saeki-fibers}	
\end{figure}

The surface fibers above the $(1,\cdot)$--strata are the second collection of subsets that filter $M$, and they are assigned the indices $(2,j)$ for $j=1,\dots, N_2$, where $N_2$ is the number of surfaces.
The boundary circles of the surfaces are each subsets of the filtration, indexed by the $(1,i)$ indices, so $(1,i)\leq (2,j)$ if and only if $M_{(1,i)}$ is one of the boundary components of $M_{(2,j)}$.
These surfaces intersect one another only when they share a boundary circle, so $(2,j)\nleq (2,k)$ for any $j,k$.

There are three types of region in the decomposition: face, edge, and vertex.
Regardless of the type of region, they are indexed in our filtration similarly to the edges.
A fiber above a region is a 3--manifold with corners formed by the $(1,i)$- and $(2,j)$-level strata, and fibers are disjoint away from their boundaries.
We therefore index fibers above regions with the indices $(3,k)$ for $k=1,\dots, N_3$ where $N_3$ is the total number of fibers above regions, put $(n,i)\leq (3,k)$ if and only if $M_{(n,i)}$ is contained in the boundary of $M_{(3,k)}$.

Recall that we call a closed 3--dimensional stratum of $M$ a \emph{block}.
At the beginning of this chapter we laid out the structure of the blocks that we expect to find inside of $M$ and, with stratification complete, we are now able to investigate these structures and determine how they are formed.
Blocks are categorized by the $\RR$-stratification regions they project to: as face, edge, or vertex blocks.
This classification determines the possible singularities that we can find inside of a block.
We first restate the definitions of the various blocks and include partial justifications based on the singularities of $M$.
In Theorem \ref{thm:block-structure} we prove that the stratification conditions impose this structure on the blocks of $M$.

{\renewcommand\labelitemi{}
	\begin{itemize}
		\item \textbf{Face block:}
		Let $B$ be a face block that fibers over the face region $F$.
		Then $B$ is stratified--homeomorphic to $S^1\times F$.
		
		\item \textbf{Edge block:}
		Let $B$ be an edge block that fibers over the edge region $E$.
		Let $A$ be the annulus $S^1\times\Ilit$ and $P$ the pair-of pants surface (i.e.\ $D^2$ minus a pair of disjoint open balls).
		If $B$ is a regular fiber over $E$ then $B$ is stratified--homeomorphic to $S^1\times E$, hence also stratified--homeomorphic to $A\times\Ilit$.
		Otherwise, $B$ is a singular fiber over $E$ and contains part of a definite or indefinite fold.
		In this case we call $B$ a \emph{definite} or \emph{indefinite edge block}.
		A definite edge block is stratified--homeomorphic to $D^2\times\Ilit$ and an indefinite edge block is stratified--homeomorphic to $P\times\Ilit$.
		
		\item \textbf{Vertex block:}
		Let $B$ be a vertex block that fibers over the vertex region $V$.
		If $B$ is a regular fiber then it is homeomorphic to $S^1\times V$, therefore homeomorphic to a (3,1)--handlebody of genus 1.
		Otherwise, we see from Figure \ref{fig:saeki-fibers} that the singular fiber above the codimension 2 singularity contained in $V$ is either connected or disconnected.
		If the singular fiber is disconnected then there is a pair of disjoint vertex blocks that each contain one of the singular fibers, hence part of a definite or indefinite fold.
		We therefore classify these blocks as \emph{definite} or \emph{indefinite vertex blocks}.
		If the singular fiber is connected, then the block containing it is an \emph{interactive vertex block}.
		A definite (resp.\ indefinite) vertex block extends and connects definite (resp.\ indefinite) edge blocks, and is homeomorphic to a (3,1)--handlebody of genus 0 (resp.\ 2).
		An interactive vertex block is homeomorphic to a (3,1)--handlebody of genus 3.
	\end{itemize}
}

\begin{rmk}
	The structures of the blocks
	%	 described in Theorem \ref{thm:block-structure}
	are roughly disk bundles over a representative fiber for the given region or, equivalently, regular neighbourhoods of that fiber.
	For a block that is a regular fiber, the representative is a circle.
	For a definite or indefinite block, the representative is the singular fiber containing a definite or indefinite fold, and for an interactive block the representative fiber is the singular fiber above the codimension 2 singular value.
\end{rmk}


\begin{theorem}
	\label{thm:block-structure}
	Let $M$ be a smooth, closed, orientable 3--manifold, let $f:M\to\RR$ be a stratifying map, suppose $\RR$ has been decomposed as in Section \ref{section:smooth-decompose} and $M$ has been stratified as in this section.
	Let $B$ be a block of $M$.
	Then $B$ is a face, edge, or vertex block if $f(B)$ is a face, edge, or vertex region respectively.
	
%	, and a block has one of the following structures:

\end{theorem}


\begin{proof}[Proof of Theorem \ref{thm:block-structure}]
	We split the proof into three parts.
	The first part proves that if a block $B$ is a regular fiber over the region $R$ then $B$ is stratified--homeomorphic to $S^1\times R$.
	In the second part, we prove that definite and indefinite blocks are stratified--homeomorphic to $D^2\times\Ilit$ or to $P\times\Ilit$ respectively.
	In the final part we discuss interactive vertex blocks, and show that they are homeomorphic to (3,1)--handlebodies of genus 3.
	Figures \ref{fig:codim-1-surfaces}-\ref{fig:vertex-block-incidence} illustrate block structures.
	
	\textbf{Part 1:}
	Let $B$ be a block over the region $R$, and suppose $B$ consists entirely of regular fibers over $R$.
	Then $(B, R, f|_B, S^1)$ has the structure of a circle bundle over $R$.
	A fiber bundle over a contractible space is trivial, so $B$ is homeomorphic to $S^1 \times R$.
	Furthermore, this homeomorphism is stratified by ensuring the strata of $B$ are mapped to the strata of $S^1 \times R$, where the stratification of $S^1 \times R$ is defined by the manifold with corners structure induced by the product topology.
%	See Figure \ref{fig:regular-blocks}.
	
%	\begin{figure}[h]
%		\centering
%		\includegraphics{figures/regular-blocks.png}
%		\caption{
%			\textbf{Regular blocks.}
%			Three types of regular blocks.
%			These are found as regular fibers over face, edge, and vertex regions.
%		}
%		\label{fig:regular-blocks}
%	\end{figure}
%	
	\begin{figure}[h!]
		\centering
		\includegraphics[width=\textwidth]{figures/codim-1-surfaces.png}
		\caption{
			\textbf{Surfaces over codimension 1 singularities.}
			$\gamma_s$ is an arc of singular values and $\gamma_t$ is an arc with endpoints $\pd\gamma_t = \{p,q\}$ that intersects $\gamma_s$ transversely at $x=\gamma_s\cap\gamma_t$.
			The three surfaces shown are the three possible cross-sectional surfaces that can project through $f$ over $\gamma_t$.
		}
		\label{fig:codim-1-surfaces}
	\end{figure}
	
	\begin{figure}[h!]
		\centering
		\includegraphics[width=\textwidth]{figures/codim-1-blocks.png}
		\caption{
			\textbf{Definite and indefinite blocks.}
			The blocks containing sections of definite and indefinite folds that project over codimension 1 singular values.
			These are found as singular fibers over edge and vertex regions.
		}
		\label{fig:codim-1-blocks}
	\end{figure}



	\textbf{Part 2:}
	Let $B$ be a definite or indefinite block over the region $R$.
	$R$ is a subset of the plane homeomorphic to $D^2$ with an arc $\gamma_s \subset X_f$ of singular values running from one of its edges to another.
	Let $\gamma_t$ be a second simple arc that crosses $\gamma_s$ transversely, and consider the cross-sectional surface obtained by $f\inv(\gamma_t)$.
	Figure \ref{fig:codim-1-surfaces} illustrates the possible surfaces containing the singular fiber over $x=\gamma_s\cap\gamma_t$.
		



	This cross section is general, so we fit a tubular neighbourhood $\nu(\gamma_s)$ about $\gamma_s$ in $R$ to obtain a bundle structure for $f\inv(\nu(\gamma_s))$ whose fiber is one of the cross-sectional surfaces (a disk or a pair-of-pants) and whose base is the arc $\gamma_s$, i.e. an interval.
	The interval is contractible, so $f\inv(\nu(\gamma_s))$ is homeomorphic to $\Sigma\times\Ilit$ for $\Sigma$ a disk or a pair-of-pants surface.
	Away from $\nu(\gamma_s)$, $R$ consists entirely of regular values so we obtain solid tori (cf. Part 1) that extend the $\Sigma\times\Ilit$ structure as seen in Figure \ref{fig:codim-1-blocks}.

	As with Part 1, the homeomorphism described is stratified by ensuring the strata of $B$ are mapped to the strata of $\Sigma \times \Ilit$, where the stratification of $\Sigma \times \Ilit$ is defined by the manifold with corners structure induced by the product topology.
	
	\textbf{Part 3:}	
	Let $B$ be an interactive block over the region $R$.
	Interactive blocks occur over octagonal vertex regions where the singular fiber above the region's codimension 2 singularity is connected, so we investigate these fibers.
	The codimension 2 singular value lies at the intersection of a pair of arcs of codimension 1 singular values.
	Call the arcs $\gamma_1$ and $\gamma_2$, let $x = \gamma_1\cap \gamma_2$, and denote the interactive singular fiber over $x$ by $B_x = B\cap f\inv(x) = \{b\in B\:|\:f(b)=x\}$.
	Our method of investigation begins by examining the possible resolutions of $B_x$ and combining those resolutions to form a genus 3 surface.
	 
	Figure \ref{fig:codim-2-interactive-fiber-1} demonstrates resolutions of the singular points of $B_x$ when $B_x$ has the first interactive singular fiber form presented in Figure \ref{fig:saeki-fibers}.
	We first note that all of the displayed fibers have inherited an orientation from $M$.
	This forces fiber resolution to be unambiguous, and allows us to identify fibers when forming the surface shown in Figure \ref{fig:codim-2-surface-1}.
	
	\begin{figure}[h!]
		\centering
		\includegraphics[width=\textwidth]{figures/codim-2-interactive-fiber-1.png}
		\caption{
			\textbf{Resolutions of the singular points in the first interactive fiber.}
			The singular fiber inside of $B_x$ and its possible resolutions over nearby codimension 1 singular values and regular values.
			The fibers inherit orientation from $M$, and this illustration is presented without loss of generality.
			This figure is modeled after Figure 18 from \cite{CostThur08}.
		}
		\label{fig:codim-2-interactive-fiber-1}
	\end{figure}

	\begin{figure}[h!]
	\centering
	\includegraphics[width=\textwidth]{figures/codim-2-surface-1.png}
	\caption{
		\textbf{Surface $\Sigma$ near the first interactive fiber that projects over $\pd\bar\nu(x)$.}
		The surface and $B_x$ are presented as embedded in $S^3$, where $H(B_x)$ is the genus-3 (3,1)--handlebody on the `inside' of $\Sigma$ in $S^3$.
	}
	\label{fig:codim-2-surface-1}
	\end{figure}

	We form the surface shown in Figure \ref{fig:codim-2-surface-1} by gluing together surfaces that project over simple arcs transversing the codimension 1 singular values.
	Gluing is performed over the boundary circles of these surfaces, and is prescribed by the resolutions in Figure \ref{fig:codim-2-interactive-fiber-1}.
	The transverse preimage containing the left column of fibers in Figure \ref{fig:codim-2-interactive-fiber-1} is a pair of pants with two green `cuffs' (top left) and a blue `waist' (bottom left).
	The preimage containing the top row is a pair of pants with two green cuffs (top left) and a pink waist (top right).
	The first gluing that helps realize the surface in Figure \ref{fig:codim-2-surface-1} is of the `left' transverse preimage surface with the `top' transverse preimage surface over their shared green cuffs.
	The surfaces recovered as transversal preimages are then:
	
%	The bottom row and left column each also produce single pairs of pants, and we continue the gluing: purple cuffs to purple cuffs from right to bottom, blue waist to blue waist from bottom to left, then green cuffs to green cuffs from left to top.
%	Figure \ref{fig:codim-2-surface-1} depicts the result of this gluing.


	{\renewcommand\labelitemi{}
	\begin{itemize}
		\item \textbf{Left:} a pair of pants with blue waist and green cuffs
		\item \textbf{Top:} a pair of pants with pink waist and green cuffs
		\item \textbf{Right:} a pair of pants with pink waist and purple cuffs
		\item \textbf{Bottom:} a pair of pants with blue waist and purple cuffs
	\end{itemize}
	}

	
	The surface $\Sigma$ in Figure \ref{fig:codim-2-surface-1} is the boundary of $H(B_x)$, a regular neighbourhood of $B_x$ in $M$, i.e. a genus-3 (3,1)--handlebody inside of $M$.
	$H(B_x)$ projects through $f$ over $\bar\nu(x)$, a closed tubular neighbourhood of $x$, and $\Sigma$ projects over $\pd(\bar\nu(x))$.
	
	Figure \ref{fig:codim-2-surface-1} presents $\Sigma$ and $B_x$ as objects embedded in $S^3$, where $\Sigma$ bounds genus-3 (3-1)--handlebodies on both sides.
	We take the `inside' component of $S^3\setminus\Sigma$ (i.e. the component containing $B_x$) to be $H(B_x)$.
	
	Outside of $\bar\nu(x)$ we use the investigations from Parts 1 and 2 of this proof. The rest of $R$, $f\inv(R\setminus\bar\nu(x))$, has the structure of a $\Sigma$-bundle over the interval, and the bundle extends $H(B_x)$ to the boundary of $R$, preserving the structure as a genus-3 (3,1)--handlebody.
	We conclude that $B$ is homeomorphic to a genus-3 (3,1)--handlebody.
	
		\begin{figure}[h!]
		\centering
		\includegraphics[width=\textwidth]{figures/codim-2-interactive-fiber-2.png}
		\caption{
			\textbf{Resolutions of the singular points in the second interactive fiber.}
			The singular fiber inside of $B_x$ and its possible resolutions over nearby codimension 1 singular values and regular values.
			The fibers inherit orientation from $M$, and this illustration is presented without loss of generality.
			This figure is modeled after Figure 16 from \cite{CostThur08}.
		}
		\label{fig:codim-2-interactive-fiber-2}
	\end{figure}
	
	\begin{figure}[h!]
		\centering
		\includegraphics[width=\textwidth]{figures/codim-2-surface-2.png}
		\caption{
			\textbf{Surface $\Sigma$ near the second interactive fiber that projects over $\pd\bar\nu(x)$.}
			The surface and $B_x$ are presented as embedded in $S^3$, where $H(B_x)$ is the genus-3 (3,1)--handlebody on the `outside' of $\Sigma$ in $S^3$.
		}
		\label{fig:codim-2-surface-2}
	\end{figure}

	An identical argument is made when $B_x$ has the second interactive singular fiber form, using Figures \ref{fig:codim-2-interactive-fiber-2} and \ref{fig:codim-2-surface-2} in place of Figures \ref{fig:codim-2-interactive-fiber-1} and \ref{fig:codim-2-surface-1} respectively.
	In this case the surfaces recovered as transversal preimages are:
	{\renewcommand\labelitemi{}
	\begin{itemize}
		\item \textbf{Left:} a pair of pants with blue waist and green cuffs
		\item \textbf{Top:} the disjoint union of a pair of pants with green waist and pink cuffs with an annulus with one green boundary circle and one pink boundary component
		\item \textbf{Right:} the disjoint union of a pair of pants with purple waist and pink cuffs with an annulus with one purple boundary circle and one pink boundary component
		\item \textbf{Bottom:} a pair of pants with blue waist and purple cuffs
	\end{itemize}
	}
	


%	Figure \ref{fig:codim-2-blocks} displays both possible interactive block structures, highlighting their boundaries.
%	The figure explains that the blocks are the handlebodies on the `outside' of the illustrated surfaces.
%	This is specifically to aid visualization of the effects of 2-- and 3--handle attachment in the next two sections, as the result of these attachments will fill the genus-3 (3,1)--handlebody on the `inside' of the illustrated surface.

%	\begin{figure}[h!]
%%		HAND DRAW THIS FIGURE
%	\caption{
%		\textbf{Possible interactive block structures embedded in $S^3$.}
%		Interactive blocks with indicated boundary stratification induced by $f\inv \pd R$.
%		Blocks are embedded in $S^3$, outside of the illustrated stratified boundary surfaces.
%	}
%	\label{fig:codim-2-blocks}
%	\end{figure}
\end{proof}

A smooth map $f$ satisfying the stratification conditions of Section \ref{section:smooth-projection} induces a decomposition on $\RR$ and a stratification of $M$.
It is important to note here that the restrictions on $f$ can induce a wide variety of possible stratifications of $M$, highlighting the variability of the resulting 4--manifold.
We end this section with a lemma that guarantees the stratified 2--handle attachments of the next section can be made over our face blocks in any order, hence we can assume all attachments occur simultaneously.


\begin{lem}
	Let $M$ be a 3--manifold with stratification induced as in Theorem \ref{thm:block-structure}.
	Then blocks of the same type (i.e. face, edge, vertex) are disjoint.
\end{lem}

\begin{proof}
	The fibers above a given region are disjoint, so the blocks above that region are disjoint.
	Regions of the same type are disjoint, hence blocks that are fibers above differing regions are also disjoint.
\end{proof}




\section{Attach Handles}

Stratifying $M$ allows the definition of attachment neighbourhoods for stratified 2--, 3--, and 4--handles in $W=M\times\Ilit$.
A 4--dimensional 2--handle is attached over a closed solid torus embedded in the boundary of a 4--manifold, so attachment neighbourhoods for our stratified 2--handles are straightforward: they are the face blocks of $M$.
We alter the boundary of $W$ by attaching handles, so the attachment neighbourhoods for 3--handles (4--handles, resp.) must be found after 2--handles (3--handles, resp.) are attached.
For 3--handles, an attachment neighbourhood consists of the union of an edge block with some strata introduced by 2--handle attachment.
For 4--handles, an attachment neighbourhood consists of the union of an edge block with some strata introduced by 2-- and 3--handle attachment.
We investigate the consequences of handle attachment by comparing the boundary of the initial manifold with the boundary of the manifold resulting from handle attachment, and use a precisely defined handle structure to focus the investigation.
Our first step is to precisely define the structure of the stratified 2--handles that we are attaching.

\subsection{2--handles}

Let $B$ be a face block of $M_1$.
By Theorem \ref{thm:block-structure}, $B$ is a closed solid torus that is stratified-homeomorphic to $S^1\times G_n$, where $G_n$ is an $n$--gon for some $n$.
Consider an unknotted embedding of $B$ in $S^3$, as illustrated in Figure \ref{fig:face-block-complement}.
The complement of the interior of $B$ is another closed solid torus $B'$, and we stratify $B'$ as follows.
First include the shared stratified boundary of $B$.
Next, introduce meridinal disks of $B'$ that are bounded by the $(1,i)$--indexed strata in $B$, i.e.\ the boundary curves in $B$ corresponding to $S^1\times c_m$ where $c_m$ is a corner of $G_n$, $m=1\dots n$.
Finally, add the homeomorphic 3--disks of $B'$ whose boundaries consist of one longitudinal annulus of $B$ along with the two meridinal disks in $B'$ that are bounded by the annulus's circular boundary components.
The filtration of $B'$ is created identically to that of the original face blocks, using inclusion as a partial ordering and indexing a stratum by its dimension as a submanifold of $B'$.
Figure \ref{fig:face-block-complement} illustrates a face block $B$ and its complement inside of $S^3$.

\begin{figure}[h!]
	\centering
	\includegraphics[width=\textwidth]{figures/face-block-complement.png}
	\caption{
		\textbf{A face block and its complement inside of $S^3$.}
		A face block $B$ is a stratified closed solid torus that is stratified-homeomorphic to $S^1\times G_n$ for some $n$--gon $G_n$.
		The complement of its unknotted interior in $S^3$ is another stratified closed solid torus $B'$.
	}
	\label{fig:face-block-complement}
\end{figure}

Taking $S_B^3$ to be the stratified $S^3$ formed as the union of $B$ and $B'$, we craft a stratified 2--handle structure as $C(S_B^3)$, the cone of $S_B^3$.
We call $C(S_B^3)$ the \emph{stratified 2--handle induced by $B$}.
Attaching $C(S_B^3)$ to $W$ over $B\subset M_1$ alters the boundary of $W$ by replacing $B\subset M_1$ with $B'$.
The full extent of this surgery can be detected by examining the edge and vertex blocks of the stratification that are incident to $B$.

\begin{theorem}
	\label{thm:primed-block-structure}
	Let $M$ be a smooth, closed, stratified, orientable 3--manifold with stratification induced by a stratifying map $f$, let $\mathfrak{B}$ be the set of face blocks of $M_1$, and let $W=M\times\Ilit$.
	Consider the 4--manifold $W'$ constructed from $W$ as
	\[
		W' = W\cup\{C(S_B^3)\}_{B\in \mathfrak{B}} / \sim,
	\]
	where $\sim$ is defined by $b\sim \iota(b)$, $\iota$ the identity map $C(S_B^3)\supset B\overset{\iota}{\to} B\subset M_1$.
	Then $M_1'=\pd W'\setminus M_0$ is a stratified 3--manifold with a decomposition into \emph{primed edge blocks} and \emph{primed vertex blocks} such that the decomposition is well-defined and the primed blocks have the following structure:
	\begin{itemize}
		\item \emph{primed edge block:}
		Let $E'$ be a primed edge block.
		Then $E'$ is identical to an edge block $E$ from Theorem \ref{thm:block-structure} with cylinders (copies of $D^2\times\Ilit$) glued over all annular boundary strata of $E$ (i.e. each closed strata $A$ of $E$ such that $A=E\cap B$ for some face block $B\in\mathfrak{B}$).
		Thus $E'$ is a stratified 3--manifold homeomorphic to $S^2\times\Ilit$
		
		\item \emph{primed vertex block:}
		Let $V$ be a primed vertex block.
		Then $V$ is identical to a vertex block from Theorem \ref{thm:block-structure} with cylinders glued over each annular boundary stratum $A$ of $V$ such that $A=V\cap B$ for some face block $B\in\mathfrak{B}$.
		Thus $V$ is a stratified 3--manifold whose boundary components are each homeomorphic to $S^2$.
	\end{itemize}
\end{theorem}

\begin{proof}
	We split the proof of this theorem into three parts.
	In the first two parts, we prove that we can find the prescribed primed edge and vertex block structures in $M_1'$.
	In the third part, we show that these structures exhaust $M_1'$.
	
	\textbf{Part 1:}
	Edge blocks occur in three possible forms: regular edge blocks as $A\times\Ilit$, definite edge blocks as $D^2\times\Ilit$, and indefinite edge blocks as $P\times\Ilit$.
	Figure \ref{fig:edge-block-incidence} displays the possible edge block forms and indicates the boundary strata of an edge block that are shared by face blocks.
	
	\begin{figure}[h!]
		\caption{
			\textbf{Edge blocks.}
			The possible edge blocks of $M_1$.
			Annular boundary strata that are incident with face blocks of $M_1$ have been indicated.
			Gluing cylinders over the indicated annuli results in a stratified 3--manifold homeomorphic to $S^2\times\Ilit$.
		}
		\label{fig:edge-block-incidence}
	\end{figure}
	
	Figure \ref{fig:edge-face-shared-boundary} shows the annulus shared by a model edge block $E$ and $B$ in $M_1$ and its corresponding annulus shared by $B$ and $B'$ in $S_B^3$.
	The figure demonstrates that the effect on $\pd W$ near $E$ of attaching $C(S_B^3)$ over $B$ is equivalent to gluing a well-defined cylinder from the stratification of $B'$  to $E$ over the annular boundary stratum $E\cap B$.
	Attachment of all 2--handles applies this cylinder gluing to all annular strata of $E$, and applying this cylinder gluing to all annular strata of $E$ results in a 3--manifold homeomorphic to $S^2\times\Ilit$.

	\begin{figure}[h!]
		\caption{
			\textbf{The effect of stratified 2--handle attachment on a model edge block.}
			A model edge block $E$, a face block $B$, its complement $B'$, and the effect of attaching $C(S_B^3)$ to $W$ over $B$ on $E$.
			The boundary stratum shared by $E$ and $B$ in $M_1$ is indicated, as is the corresponding boundary stratum of $B'$ in $S_B^3$.
		}
		\label{fig:edge-face-shared-boundary}
	\end{figure}
	
	\textbf{Part 2:}	
	Vertex blocks occur in five possible forms: regular vertex blocks as $S^1$-bundles over the vertex region, indefinite and definite vertex blocks stratified-homeomorphic to those that appear in Figure \ref{fig:codim-1-blocks}, and interactive vertex blocks stratified-homeomorphic to those that appear in Figure \ref{fig:codim-2-blocks}.
	Figure \ref{fig:vertex-block-incidence} displays the possible vertex block forms embedded in $S^3$ and indicates the boundary strata of a vertex block that are shared by face blocks.
	
	\begin{figure}[h!]
		\caption{
			\textbf{Vertex blocks.}
			The possible vertex blocks of $M_1$.
			Annular boundary strata that are incident with face blocks of $M_1$ have been indicated.
			Gluing cylinders over the indicated annuli results in a stratified 3--manifold whose boundary is a collection of stratified 2--spheres.
		}
		\label{fig:vertex-block-incidence}
	\end{figure}	
	
	Figure \ref{fig:vertex-face-shared-boundary} displays a model vertex block $V$ embedded on the `outside' of its stratified genus-3 boundary surface in $S^3$, and shows the annulus shared by $V$ and $B$ in $M_1$ and its corresponding annulus shared by $B$ and $B'$ in $S_B^3$.
	The figure demonstrates that the effect on $\pd W$ near $V$ of attaching $C(S_B^3)$ over $B$ is equivalent to gluing a well-defined cylinder from the stratification of $B'$  to $V$ over the annular boundary stratum $V\cap B$.
	Attachment of all 2--handles applies this cylinder gluing to all annular strata of $V$ shared by face blocks, and applying this cylinder gluing to such annular strata of $V$ results in a 3--manifold whose boundary components are each homeomorphic to $S^2$.
	Moreover, these boundary components are exactly the boundary components of the primed edge blocks.
	
	\begin{figure}[h!]
		\caption{
			\textbf{The effect of stratified 2--handle attachment on a model vertex block.}
			A model edge block $V$, a face block $B$, its complement $B'$, and the effect of attaching $C(S_B^3)$ to $W$ over $B$ on $V$.
			The boundary stratum shared by $V$ and $B$ in $M_1$ is indicated, as is the corresponding boundary stratum of $B'$ in $S_B^3$.
		}
		\label{fig:vertex-face-shared-boundary}
	\end{figure}

	\textbf{Part 3:}
	To show that primed edge and vertex blocks exhaust $M_1'$, we show that the entirety of $B'$ has been apportioned among the primed blocks for each $B'$.
	The $(3,k)$ strata of $B'$ are each cylinders whose annular boundary strata corresponds directly to a longitudinal boundary annulus of $B$.
	Such an annulus projects through $f$ to a boundary edge of a face region.
	Face regions do not intersect even on their boundary, so all of the annular boundary strata of $B$, hence $B'$, are shared only by edge and vertex blocks.
	Thus each cylinder of $B'$ has been assigned to a primed edge or vertex block.
	Because 2--handle attachment altered $M_1$ to $M_1'$ only by replacing each face block $B$ with its complementary $B'$, the primed blocks must exhaust $M_1'$.
\end{proof}

Decomposing $M_1'$ into primed edge and vertex blocks provides us with stratified 3--handle attachment neighbourhoods.
A 4--dimensional 3--handle is attached over an $S^2\times\Ilit$ embedded in the boundary of a 4--manifold, so attachment neighbourhoods for our stratified 3--handles are the primed edge blocks of $M_1'$.
We begin 3--handle attachment by precisely defining the structure of a stratified 3--handle so that it may be attached over a primed edge block.

\subsection{3--handles}

Let $E'$ be a primed edge block of $M_1'$.
By Theorem \ref{thm:primed-block-structure}, $E'$ is homeomorphic to $S^2\times\Ilit$.
In particular, $\pd E'$ is a disjoint pair of stratified 2--spheres.
We form a 4--disk containing $E'$ in its boundary by a double coning method on $E'$: we first cone the spherical boundary components of $E'$ to form a 3--disk, then cone the 3--disk to obtain a 4--disk.


For each stratified boundary sphere $S_i^2\in\pd E'$ we form the stratified 3--disk $C(S_i^2)$ and glue $C(S_i^2)$ to $E'$ over $S_i^2$.
The result is a stratified 3--disk, and we further cone that 3--disk to form a stratified 4--disk.
We denote the 4--disk by $C^2(E')$ and call the resulting stratified 3--handle structure the \emph{stratified 3--handle induced by $E'$}.
Attaching $C^2(E')$ to $W'$ over $E'\subset M_1'$ alters the boundary of $W'$ by replacing $E'$ with the 3--disks $C(S_i^2)$ glued over the corresponding stratified boundary spheres in $M_1'$.
The full extent of this surgery can be detected by examining the primed vertex blocks of $M_1'$.

\begin{cor}
	\label{thm:primed-primed-block-structure}
	Let $W'$ be the 4--manifold resulting from the construction described in Theorem \ref{thm:primed-block-structure} and let $\mathfrak{E}'$ be the set of primed edge blocks of $M_1'\subset\pd W'$.
	Consider the 4--manifold $W''$ constructed from $W'$ as
	\[
	W'' = W'\cup\{C^2(E')\}_{E'\in \mathfrak{E}'} / \sim,
	\]
	where $\sim$ is defined by $e\sim \iota(e)$, $\iota$ the identity map $C^2(E')\supset E'\overset{\iota}{\to} E'\subset M_1'$.
	Then $M_1''=\pd W''\setminus M_0$ is a stratified 3--manifold with a decomposition into \emph{double primed vertex blocks} $V''$ such that the decomposition is well-defined and each $V''$ is homeomorphic to $S^3$.
\end{cor}

\begin{proof}
	For each vertex block $V$, the cylinders introduced by 2--handle attachment and the 3--disks introduced by 3--handle attachment combine into a handlebody $H$ of the same genus as $V$.
	$V$ and $H$ combine to form each component of $M_1''$, and we show a Heegaard diagram of complementary meridinal circles for each case in Figures \ref{fig:heegaard-genus-0}-\ref{fig:heegaard-genus-3-2}.
	Thus $M_1''$ consists of a collection of stratified 3--spheres \cite{SchlWald}.
	
	\begin{figure}[h!]
		\caption{
			\textbf{Heegaard system for genus-0 vertex blocks.}
			$V$ is a genus-0 handlebody when $V$ is a definite singular fiber of $f$.
			There is only one possible Heegard diagram for genus-0 handlebodies, and the result is trivially $S^3$.
		}
		\label{fig:heegaard-genus-0}
	\end{figure}

	\begin{figure}[h!]
		\caption{
			\textbf{Heegaard system for genus-1 vertex blocks.}
			$V$ is a genus-1 handlebody when $V$ is a regular fiber of $f$.
			The Heegaard diagram consists of a meridian of each of $V$ and $H$.
		}
		\label{fig:heegaard-genus-1}
	\end{figure}

	\begin{figure}[h!]
		\caption{
			\textbf{Heegaard system for genus-2 vertex blocks.}
			$V$ is a genus-2 handlebody when $V$ is an indefinite singular fiber of $f$.
			The Heegaard diagram for $H$ consists of the pants cuffs, and for $V$ the side seam.
			% we use the cuffs and the outseams from one pair of pants to form the heegaard diagram
		}
		\label{fig:heegaard-genus-2}
	\end{figure}

	
	\begin{figure}[h!]
		\caption{
			\textbf{Heegaard system for genus-3 vertex blocks, case 1.}
			$V$ is a genus-3 handlebody when $V$ contains an interactive singular fiber.
			There are two types of interactive singular fibers, we show here the first from Figure \ref{fig:saeki-fibers}
		}
		\label{fig:heegaard-genus-3-1}
	\end{figure}
	
	
	\begin{figure}[h!]
		\caption{
			\textbf{Heegaard system for genus-3 vertex blocks, case 2.}
			$V$ is a genus-3 handlebody when $V$ contains an interactive singular fiber.
			There are two types of interactive singular fibers, we show here the second from Figure \ref{fig:saeki-fibers}
		}
		\label{fig:heegaard-genus-3-2}
	\end{figure}	
\end{proof}

After attaching 3--handles over primed edge blocks, $W''$ is a 4--manifold with boundary consisting of $M_0$ and a collection of stratified 3--spheres.
We attach stratified 4--handles over these 3--spheres, so we begin by precisely defining the structure of these handles.

\subsection{4--handles}
Let $V''$ be a double primed vertex block of $M_1''$.
By Theorem \ref{thm:primed-primed-block-structure}, $V''$ is homeomorphic to $S^3$.
We form a 4--disk whose boundary is $V''$ by taking the cone of $V''$.
We denote this 4--disk by $C(V'')$ and call the resulting 4--handle structure the \emph{stratified 4--handle induced by $V''$}.
Attaching $C(V'')$ to $W''$ over $V''$ alters the boundary of $V''$ by replacing it with $\emptyset$.

\begin{cor}
	Let $W''$ be the 4--manifold resulting from the construction described in Theorem \ref{thm:primed-primed-block-structure} and let $\mathfrak{V}''$ be the set of double primed vertex blocks of $M_1''\subset\pd W''$.
	Consider the 4--manifold $W'''$ constructed from $W''$ as
	\[
	W''' = W''\cup\{C(V'')\}_{V''\in \mathfrak{V}''} / \sim,
	\]
	where $\sim$ is defined by $v\sim \iota(v)$, $\iota$ the identity map $C(V'')\supset V''\overset{\iota}{\to} V''\subset M_1''$.
	Then $M_1'''=\pd W'''\setminus M_0=\emptyset$, hence $W'''$ is a 4--manifold whose boundary is exactly $M$.
\end{cor}




%\section{Conclusion}

This chapter presents a proof that smooth, closed, orientable 3--manifolds bound 4--manifolds by providing a method of construction.
The resulting 4--manifold is not smooth, but it is stratified and the stratification is well-defined.
Moreover, the 4--manifold depends heavily on the map $f:M\to\RR$, and no investigation is made into any relationship between 4-manifolds produced by $f,g:M\to\RR$ as would be explained by a relationship between $f$ and $g$.




