\section{Projections from 3--manifolds to $\RR$}
\label{section:smooth-projection}

Our stratification of $M$ is induced by a decomposition of the plane, itself induced by the singular values of a smooth map $M\to\RR$.
To prove that a stratification suitable for our construction exists for any smooth orientable 3--manifold, we show first that an inducing decomposition of $\RR$ exists.
To prove that such a decomposition of $\RR$ exists, we present the properties of $f:M\to\RR$ required to induce the decomposition, and argue why a map possessing such properties exists for any smooth, orientable 3--manifold.

Let $f:M\to\RR$ be a smooth map, let $df$ be the differential of $f$, and let $S_r(f)$ be the set of points in $M$ such that $df$ has rank $r$.
Then we require that the following be true of $f$:
\begin{enumerate}
	\item $S_0(f)$ is empty.
	
	\item $S_1(f)$ consists of smooth non-intersecting curves.  We call these the \emph{fold curves} of $f$.
	
	\item The set of points where $f|_{S_1(f)}$ has zero differential (these can appear in the image of $S_1(f)$ as cusps in the plane) is empty.
	
	\item Let $\gamma_i, \gamma_j, \gamma_k \in S_1(f)$ be fold curves. Then $f(\gamma_{i,j,k})$ are submanifolds of $\RR$ such that
	\begin{enumerate}
		\item $f(\gamma_i)$ and $f(\gamma_j)$ intersect transversely,
		
		\item $f(\gamma_i)\cap f(\gamma_j)\cap f(\gamma_k)$ is empty (i.e. there are no triple-intersections), and
		
		\item self-intersections of $f(\gamma_i)$ are transverse
	\end{enumerate}
	
	\item\label{strat-con:fold-type} If $p\in S_1(f)$ then there exist coordinates $(u, z_1, z_2)$ centred at $p$ and $(x,y)$ centred at $f(p)$ such that $f$ takes the form of either
	\begin{enumerate}
		\item\label{property:paraboloid} $(x,y)=(u,\pm(z_1^2+z_2^2))$, or
		\item\label{property:saddle} $(x,y)=(u,\pm(z_1^2-z_2^2))$
	\end{enumerate}
	in a neighbourhood of $p$.
	If $f$ takes the form of \ref{property:paraboloid} then we further classify $p$ as a \emph{definite fold}, and if $f$ takes the form of \ref{property:saddle} then $p$ is an \emph{indefinite fold}.
	
	\item The set of singular values of $f$ in the plane is connected
	
	\item $f(M)$ is bounded.
\end{enumerate}

The first four conditions ensure that the set of singular values in the plane is well-behaved, the fifth allows classification of singular fibers of $f$ using the work of Saeki \cite{Saeki}, and the sixth guarantees that the set of singular values of $f$ partitions the image of $M$ into discs.
We call these five conditions the \emph{stratification conditions} on $f$.