\section{Projections from 3--manifolds to $\RR$}
\label{section:smooth-projection}

Our stratification of $M$ is induced by a decomposition of the plane, itself induced by the singular values of a smooth map $M\to\RR$.
To prove that a stratification suitable for our construction exists for any smooth orientable 3--manifold, we show first that an inducing decomposition of $\RR$ exists.
To prove that such a decomposition of $\RR$ exists, we present the properties of $f:M\to\RR$ required to induce the decomposition, and argue why a map possessing such properties exists for any smooth, orientable 3--manifold.

Let $f:M\to\RR$ be a smooth map, let $df$ be the differential of $f$, and let $S_r(f)$ be the set of points in $M$ such that $df$ has rank $r$.
Then we require that the following be true of $f$:
\begin{enumerate}
	\item $S_0(f)$ is empty.
	
	\item $S_1(f)$ consists of smooth non-intersecting curves.  We call these the \emph{fold curves} of $f$.
	
	\item The set of points where $f|_{S_1(f)}$ has zero differential is empty.
	
	\item\label{strat-con:fold-type} If $p\in S_1(f)$ then there exist coordinates $(u, z_1, z_2)$ centred at $p$ and $(x,y)$ centred at $f(p)$ such that $f$ takes the form of either
	\begin{enumerate}
		\item\label{property:paraboloid} $(x,y)=(u,\pm(z_1^2+z_2^2))$, or
		\item\label{property:saddle} $(x,y)=(u,\pm(z_1^2-z_2^2))$
	\end{enumerate}
	in a neighbourhood of $p$.
	If $f$ takes the form of (\ref{property:paraboloid}) then we further classify $p$ as a \emph{definite fold}, and if $f$ takes the form of (\ref{property:saddle}) then $p$ is an \emph{indefinite fold}.
	
	\item Let $\gamma_i, \gamma_j, \gamma_k \in S_1(f)$ be fold curves. Then each of $f(\gamma_i)$, $f(\gamma_j)$, $f(\gamma_k)$ is a submanifold of $\RR$ such that
	\begin{enumerate}
		\item $f(\gamma_i)$ and $f(\gamma_j)$ intersect transversely,
		
		\item $f(\gamma_i)\cap f(\gamma_j)\cap f(\gamma_k)$ is empty (i.e. there are no triple-intersections), and
		
		\item self-intersections of $f(\gamma_i)$ are transverse
	\end{enumerate}
	
	\item The set of singular values of $f$ in the plane is connected
	
	\item The image of $M$ through $f$ is bounded in the plane.
	
% f is continuous and M is compact so f(M)is compact in R^2 (ie is closed and bounded)
\end{enumerate}

%The first four conditions ensure that the set of singular values in the plane is well-behaved, the fifth allows classification of singular fibers of $f$ using the work of Saeki \cite{Saeki}, and the sixth guarantees that the set of singular values of $f$ partitions the image of $M$ into discs.
We call these the \emph{stratification conditions} on $f$, and we call a map satisfying the stratification conditions a \emph{stratifying map}.
%Each stratification condition simplifies the arguments of this chapter in some way.
%The first three conditions simplify the description of $X_f=f(S(f))$, ensuring that $X_f$ is describable as a collection of smooth closed curves in the plane.
%The fourth condition simplifies the discussion of neighbourhoods of singular fibers of $f$.
%Condition 5 ensures that crossings of $X_f$ occur uniformly, simplifying the plane decomposition of Section \ref{section:smooth-decompose}.
%The final two conditions ensure that $\RR\setminus X_f$ consists of a finite collection of open surfaces $\Sigma_i$, $i=1\dots k$ each homeomorphic to a 2--disk, and along with $\Sigma_\infty$ which is unbounded.
%Furthermore, each $\Sigma_i$ is contained in the image of $f$ whereas $\Sigma_\infty\cap f(M)$ is empty.
The existence of smooth maps satisfying conditions 1--4 is discussed in \cite{Levine65}, 5 \& 6 can be guaranteed by a suitable modification of a map that satisfies 1--4, and condition 7 is always satisfied because the image of a compact set through a continuous map is compact.
A smooth map satisfying conditions 1--4 can be smoothly perturbed inside of a tubular neighbourhood of $S_1(f)$ to satisfy condition 5.
A map satisfying conditions 1--5 is homotopic to one that also satisfies condition 6, but the homotopy passes through functions that have cusps, i.e. that do not meet condition 1.

Let $K$ and $L$ be folds in $M$ such that $f(K)$ and $f(L)$ belong to disjoint connected components of $f(S(f))$ in the plane.
We create a definite fold and an indefinite fold that meet at a pair of cusp points (As in Lemma 3.1 of \cite{saeki1995constructing}).
This is a `matching pair' of cusps (due to results in section 4 of \cite{Levine65}), and we connect that matching pair of cusps using an arc that links each of $K$ and $L$ in $M$.
A band operation (Lemma 3.7 of \cite{saeki1995constructing}) on that arc then eliminates the cusps and leaves behind two folds, one definite and one indefinite, whose images in the plane necessarily intersect each of $f(K)$ and $f(L)$, and each of the first four conditions are still satisfied by these moves.

% 1-4 from lvine65
% 5-7 argue.

%Before stratifying the plane with a stratifying map, let's discuss some of the implications of the stratification conditions.
To see the main implication of condition 6, let $f:M\to\RR$ be a stratifying map and let $X_f = f(S(f))$, the set of singular values of $f$.
We first consider $f(M)\setminus X_f$.
Because $X_f$ is a connected collection of arcs in the plane that intersect only transversely, $f(M)\setminus X_f$ is a collection of connected regions in the plane, each of which consists entirely of regular values.
Stratification condition 6 guarantees that each of these regions is simply connected --- the boundary components of these regions are formed by the singular values of $f$ in the plane, and different boundary components are necessarily disjoint.
The sixth stratification condition does not restrict the class of maps we consider because, as we have just seen, a map satisfying the first five stratification conditions is smoothly homotopic to one satisfying the sixth.

%Our 4--manifold construction assumes that there is at least one arc intersection in $X_f$.
%Fortunately, there is only one example to consider when this is not the case: when $X_f$ is a simple closed curve in the plane.

