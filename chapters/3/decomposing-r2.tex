\section{Decomposing $\RR$}

Let $f:M\to\RR$ be a smooth map possessing the stratification conditions of Section \ref{section:smooth-projection} and let $X_f = f(S(f))$, the set of singular values of $f$.
$X_f$ is a connected collection of arcs in the plane that intersect only transversely.

We fit closed neighbourhoods (\emph{sleeves}) around the singular values of $f$ and classify these sleeves by the maximum codimension (with respect to $\RR$) of singular values they contain.
Because $X_f$ consists of codimension 1 and codimension 2 singular values (i.e.\ arcs and arc-crossings respectively), we decompose $\RR$ into face regions that contain no singular values, edge regions that contain only codimension 1 singular values, and vertex regions, each of which contain exactly 1 codimension 2 singular value.
Figures \ref{fig:vertex-sleeve}-\ref{fig:face-sleeve} are used to illustrate the decomposition resulting from sleeve-fitting.

\begin{figure}[h!]
	\label{fig:vertex-sleeve}
	\caption{
		\textbf{Forming vertex regions.}
		Octagonal sleeves are fit around codimension 2 singular values to form vertex regions.
	}
\end{figure}

We begin by fitting octagonal sleeves around codimension 2 singular values as in Figure \ref{fig:vertex-sleeve}.
Octagons are used here solely to simplify descriptions further down the line of proof.

Let $x$ be a codimension 2 singular value.
$x$ is the result of an arc crossing, and a small neighbourhood around an arc crossing is divided into four regions of regular values.
The octagon around $x$ is fit so its edges alternate between being fully contained in a region of regular values and transversely intersecting one of the arcs of singular values that creates $x$.
See Figure \ref{fig:vertex-sleeve} for a model fitting.

The interiors of the octagons along with the octagonal boundaries form the vertex regions of this decomposition.
The octagons are chosen to be small enough that no two vertex regions overlap

\begin{figure}[h!]
	\label{fig:edge-sleeve}
	\caption{
		\textbf{Forming edge regions.}
		Vertex region corners are connected to fit sleeves around arcs of codimension 1 singular values to form edge regions.
	}
\end{figure}



\begin{figure}[h!]
	\label{fig:face-sleeve}
	\caption{
		\textbf{Forming face regions.}
		All remaining regions contain no singular values, and we take these to be the face regions.
	}
\end{figure}
