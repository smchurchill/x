\startchapter{Discussion}
\label{chapter:conclusion}

The main chapters of this thesis serve different roles in the greater mathematical ecosystem.
It is evident from the discussion in Section 2.2 of \cite{CostThur08} that the ideas behind the constructive proof in Chapter \ref{chapter:smooth} is not itself novel, but the details had yet to be published.
Chapter \ref{chapter:smooth} fills a hole in the current literature, whereas the contents of Chapter \ref{chapter:triangulation} are entirely new, serving as a solid foundation for future research in computational topology.

The immediate next step from creation of the algorithm is an implementation for a topology software project such as Regina.
Such an implementation furthers progress toward algorithmic computation of the Rokhlin invariant for 3--manifolds and serves as another tool in generating 4--manifold censuses.
In its current form, the construction algorithm guarantees no topological features of the constructed 4--manifold save that its boundary is the given input 3--manifold.
Lucrative avenues of exploration would therefore include an investigation into invariants (e.g.\ the fundamental group) or constructs (e.g.\ spin structures) on the constructed manifold.

This work was initially an attempt to adapt the Turaev Reconstruction Theorem as an algorithm on 3--manifolds triangulations.
The theorem was introduced by Turaev \cite{Turaev91} and is also covered by Costantino \cite{Cost05}.
The algorithm would build on the process outlined in Chapter 4 of Costantino and Thurston \cite{CostThur08}, where a quadratic bound is achieved for the number of simplices needed to triangulate a 4--manifold constructed using the reconstruction theorem.

Such an adaptation necessarily passes through the shadow realm.
The shadows explored in \cite{Turaev91} are 2--complexes with strict structural restrictions imposed upon them.
In the reconstruction theorem, a shadow serves as a set of instructions for a handle decomposition of the desired 4--manifold.
Each 0--cell prescribes a (4,0)--handle, each 1--cell a (4,1)--handle, and each 2--cell a (4,2)--handle.
The resulting 4--manifold $W$ has boundary exactly the original 3--manifold $M$.
Two major obstructions eventually caused this adaptation attempt to be scrapped:
\begin{enumerate}
	\item Precisely defining (4,2)--handle attachment sites in this process is hard.
	\item The fact that $\pd W = M$ at the end of this process is not obvious and requires a deep examination of shadows (briefly, $\pd W = M$ because $M$ and $W$ have the same shadow).
\end{enumerate}

Instead, the results of this work were obtained by turning the reconstruction theorem on its head: we begin by making the precise definition of (4,2)--handle attachment sites as simple as possible.
Once (4,2)--handles have been attached, attachment of (4,3)-- and (4,4)--handles is straightforward.
This method is dual to the Turaev Reconstruction Theorem in its approach and handle indexing ($(n,\lambda)$-- and $(n,n-\lambda)$--handles are identical, just attached over complementary portions of the handle's boundary), but the goals are different.
Turaev's work is focused on studying the topology of the shadows themselves, where our ultimate goal is algorithmic construction of 4--manifolds with prescribed boundary.

Beyond its raw results, this research is a demonstration of adapting an existence proof into a constructive proof.
More adaptations of existence proofs should not be surprising as the bounds of scientific computability are pushed.
