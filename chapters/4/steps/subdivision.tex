\section{Induce subdivision}
\label{section:pl-subdivide}

The goal of subdividing $N$ is to create and identify handle attachment sites analogous to the face, edge, and vertex blocks of Chapter \ref{chapter:smooth}.
We use a similar technique to that found in Chapter \ref{chapter:smooth}, iteratively subdividing the tetrahedra of $N$ by certain preimages of $f$.
Tetrahedron subdivisions are compatible, i.e.\ tetrahedron subdivisions fit together exactly as the undivided tetrahedra do inside of $N$.

Let $\sigma$ be a tetrahedron of $N$ and let $s$ be a line segment in the plane such that $s\cap f(\sigma)$ is nonempty, the endpoints of $s$ are outside of $f(\sigma)$,  and $s\cap f(\sigma)$ is a line segment in $f(\sigma)$ disjoint from any vertices or crossings of $f(\sigma)$.
Figure \ref{fig:standard-position-intersection} demonstrates the possible configurations of these line segments, and shows that their preimages inside of $\sigma$ are triangles and quads.
We refer to these preimages as \emph{exterior triangles} and \emph{exterior quads}.
When a pair of these preimages intersect, the intersection is a line segment with endpoints interior to a pair of triangles of $\sigma^2$.

\begin{figure}[h!]
	\centering
	\includegraphics[width=0.9\textwidth]{figures/standard-position-intersection.png}
	\caption{
		\textbf{A tetrahedron $\sigma$ in standard position, intersecting edges, and preimage triangles and quads.}
		An intersecting edge separates the vertices of $\sigma$.
		If one is separated from the other three, its preimage is a triangle.
		If the vertices are separated into two pairs of two, the preimage is a quad.
	}
	\label{fig:standard-position-intersection}
\end{figure}

For each $\sigma\in N^3$ and each edge $e$ of $N^1$ such that $e\notin\sigma^1$, $f(e)$ is a line segment in the plane that is either disjoint from $f(\sigma)$ or induces an exterior triangle or quad in $\sigma$.
There are two edges of $\sigma$ whose preimages in $\sigma$ form \emph{interior triangles} and these are the \emph{interior edges} of $\sigma$.
The image of $\sigma$ is a quadrilateral, and four of $\sigma$'s edges map to the boundary of that quadrilateral.
The remaining two edges are the interior edges.
%These are the preimages of the edges of $\sigma$ that map through $f$ across $f(\sigma)$ as in Figure \ref{fig:standard-position}, and we refer to these edges as \emph{interior edges}.
%The three edges of the interior triangle induced by $e$ are $e$ itself along with a pair of edges that bisect the triangles of $\sigma$ not containing $e$ as an edge.
%The three vertices of the interior triangle induced by $e$ are the two vertices of $\sigma$ that are the endpoints of $e$ along with a third vertex located in the edge of $\sigma$ opposite $e$.
Figure \ref{fig:standard-position-interior-exterior} shows one interior triangle along with its possible intersections with exterior triangles and quads.
The interior triangles of $\sigma$ always intersect in a line segment with endpoints inside of the interior edges of $\sigma$, shown in Figure \ref{fig:standard-position-interior-interior}

\begin{figure}[h!]
	\centering
	\includegraphics[width=0.9\textwidth]{figures/standard-position-interior-exterior.png}
	\caption{
		\textbf{A tetrahedron $\sigma$ in standard position, one interior triangle, one exterior triangle, and one exterior quad.}
		There are two special preimage triangles in $\sigma$, called \emph{interior triangles}, that occur as the preimages of the edges of $\sigma$ that map through $f$ across $f(\sigma)$ as in Figure \ref{fig:standard-position}.
	}
	\label{fig:standard-position-interior-exterior}
\end{figure}

\begin{figure}[h!]
	\centering
	\includegraphics[width=0.9\textwidth]{figures/standard-position-interior-interior.png}
	\caption{
		\textbf{A tetrahedron $\sigma$ in standard position with both interior triangles.}
		As in Figure \ref{fig:standard-position-interior-exterior}, but displaying both interior triangles.
	}
	\label{fig:standard-position-interior-interior}
\end{figure}

Recall that the goal of this subdivision is to identify combinatorial analogues for the face, edge, and vertex blocks of Chapter \ref{chapter:smooth}.
If we were to subdivide $N$ using the quads and triangles induced by the edges of $N$, such blocks are ill-defined.
We amend this by introducing a set of line segments analogous to the sleeves of Section \ref{section:smooth-decompose}.

The vertices of $N$ are sent to the edge of the boundary of the disc in the plane, and this boundary is the only location in the plane that our smooth singularity theory does not apply.
Our first set of line segments remedies this.
Let $p$ be a vertex of $N$ and $s$ a secant in the disc that separates $f(p)$ from $f(N^0\setminus{p})$.
Then $f\inv(s)$ is a triangulated 2--sphere, and is the boundary of a triangulated 3--disc centred at $p$.
We use such 3--discs as our first collection of combinatorial vertex blocks

The vertices of $N$ are mapped to the $k\nth$ roots of unity and the rest of the simplices of $N$ are mapped to the plane via linear extension.
Consider the regular $k$--gon $G_k$ with vertices the $k\nth$ roots of unity reflected through the origin.
Because $k$ is odd, each of these segments is a secant in the disc that separates the image of one vertex of $N$ from the rest.
Moreover, $G_k$ separates $f(N^0)$ from the interior edge-edge crossings of $f(N)$.
We use $G_k$ as our first set of line segments.

Next, we introduce segments that induce the edge and face blocks and the rest of the vertex blocks.
For each edge $e$ of $N^1$, consider $e_\varepsilon^+ = s(e+\varepsilon_e e_\perp)$ and $e_\varepsilon^- =s(e-\varepsilon_e e_\perp)$, secants in $G_k$ that are parallel to $f(e)$ and located a small orthogonal distance away from $f(e)$.
Here we are using $s(\cdot)$ as a function that extends and trims line segments in the plane into secants in $G_k$.
We require that our $\varepsilon$'s are small enough that if $e,g\in N^1$ share a boundary vertex then $e_\varepsilon^\pm$ and $g_\varepsilon^\pm$ do not intersect (i.e. $G_k$ truncates the segments so that their intersection occurs outside of $G_k$).
%the rectangle $R_e$ in the plane defined by $e_\varepsilon^+$ and $e_\varepsilon^-$ does not fully contain $g_\varepsilon^\pm$ for any $g\in N^1$.
%This requirement also ensures that the only crossings of $f(N^1)$ contained in $R_e$ are crossings involving $e$.
The segments $e_\varepsilon^\pm$ for each $e\in N^1$ are called \emph{sleeve segments}, and their preimages form exterior triangles and quads inside of the tetrahedra of $N$.
Figure \ref{fig:pl-segments} depicts the inclusion of $G_k$ and all sleeve segments for a mapping of a triangulated $S^3$ to the plane.
Here, the triangulation of $S^3$ used is the 2-tetrahedron triangulation.

\begin{figure}[h!]
	\centering
	\includegraphics[width=0.9\textwidth]{figures/pl-sleeves.png}
	\caption{
		\textbf{The process of including additional segments into the plane to form piecewise-linear sleeves.}
		We first add the regular polygon $G_5$, which forms vertex sleeves around the image of each vertex of $N$.
		Then, sleeve segments are included as secants of $G_5$ to form face and edge sleeves as well as vertex sleeves around each interior edge-edge crossing of $N$.
	}
	\label{fig:pl-segments}
\end{figure}


%\begin{figure}[h!]
%	\centering
%	\includegraphics[width=0.9\textwidth]{figures/pl-sleeves.png}
%	\caption{
%		\textbf{A pair of sleeve segments in the plane.}
%		We show a pair of edges projections $f(e)$ and $f(g)$ in the plane, along with their associated sleeve segments.
%		Sleeve segments are necessary to ensure well-defined face, edge, and vertex block analogues in the subdivision of $N$.
%	}
%	\label{fig:pl-sleeves}
%\end{figure}
%
%\begin{figure}[h!]
%	\centering
%	\includegraphics[width=0.9\textwidth]{figures/pl-regions.png}
%	\caption{
%		\textbf{Regions in the plane that correspond to combinatorial vertex, edge, and face block.}
%		The sleeve segments provide the same functionality as the sleeves in Chapter \ref{chapter:smooth}, in that the preimages of the regions they form behave similarly to the vertex, edge, and face blocks defined in Chapter \ref{chapter:smooth}.
%	}
%	\label{fig:pl-regions}
%\end{figure}

Take $G$ to be the set of line segments in the plane that consists of $f(e)$ for each $e\in N^1$ along with the sleeve segments $e_\varepsilon^\pm$.
For each tetrahedron $\sigma$ of $N$, the triangles, quads, and intersections of $f\inv G\cap\sigma$ define a subdivision of $\sigma$ into a cell complex.
We implement this subdivision in Algorithm \ref{alg:subdividing-manifold} by constructing a new cell complex that is equivalent to $N$.
The subdivision iterates over all of the triangles and quads inside of $\sigma$, and the subdivision induced by a polygon is fully realized before moving onto the next.

For an example of subdivision, refer to Figure \ref{fig:standard-position-interior-exterior}, where a tetrahedron is being subdivided by an exterior triangle, an interior triangle, and an exterior quad.
We construct the subdivision of $\sigma$ as a 3--complex $\tau$, starting by setting $\tau = \sigma$.
Starting at the loop on line 6 of Algorithm \ref{alg:subdividing-manifold}, we set $g$ to be the brown segment that separates the pink vertex from the rest of $f(\sigma)$, making $\delta$ the brown exterior triangle in $\sigma$.
Then the loop on line 8 loops over all 1--cells of $\tau$ that intersect the exterior triangle $\delta$ which are the three 1--cells of $\tau$ incident with the pink vertex.
We subdivide these 1--cells by their intersection with $\delta$, which amounts to including the corners of the $\delta$ exterior triangle into the complex, splitting the relevant 1--cells into a pair of 1--cells on either side of the corner.
We move to the loop on line 12, where we subdivide the three boundary pentagons (formerly triangles) of $\tau$ that contain the pink vertex by the three triangular edges of $\delta$.
Each pentagon is subdivided into a triangle and a quadrilateral.
Finally, the loop at line 16 we subdivide the only 3--cell of $\tau$ into a tetrahedron containing the pink vertex and a triangular cylinder away from the pink vertex.

Heading back to the top of the loop at line 6, we set $g$ to be the blue interior edge of Figure \ref{fig:standard-position-interior-exterior} hence $\delta$ to be the blue interior triangle.
Moving through the loop at line 8, we find
\begin{enumerate}
	\item the intersection of $\delta$ with the 1--cell of $\tau$ containing the blue and red vertices and 
	\item the intersection of $\delta$ with the brown exterior triangle located on the shared boundary of the pair of 2--cells in $\pd\tau$ induced by the subdivision of the 2--cell in $\pd\sigma$ opposite the green vertex.
\end{enumerate}
We then insert 0--cells into $\tau$ at these intersections, subdividing 1--cells as necessary.
The loop at line 12 now runs over the intersections of $\delta$ with the 2--cells of $\tau$, and these intersections occur with the triangle in $\pd\tau$ opposite the pink vertex and with the 2--cells in $\pd\tau$ induced by the subdivision of the 2--cell in $\pd\sigma$ opposite the green vertex.
We add 1--cells to $\tau$ that divide the quadrilateral into a pair of quadrilaterals and the triangles into pairs of triangles.
Finally, the loop at line 16 subdivides both 3--cells of $\tau$, splitting them each in half as polyhedrons.
It is left to the reader to continue the subdivision of $\tau$ in Figure \ref{fig:standard-position-interior-exterior}, where the remaining exterior quad subdivides five 1--cells, five 2--cells, and two 3--cells.

Observe also that the gluing defined on the tetrahedra of $N$ is used to produce $M$ as well.
To see this, let $\sigma$ and $\sigma'$ be tetrahedra of $N$ that are glued together over a triangular face, $\Delta\in\sigma_2$ and $\Delta'\in\sigma'$.
Then Algorithm \ref{alg:subdividing-manifold} produces a pair of 3--complexes $\tau$ and $\tau'$, and $\Delta$, $\Delta'$ are subdivided equivalently inside of $\pd\tau$ and $\pd\tau'$, so the gluing that defines $M$ is constructed by subdividing the gluing that defined $N$.

\begin{algorithm}[h!]
	\caption{Subdividing $N$}
	\label{alg:subdividing-manifold}
	\KwData{A closed, orientable 3--manifold triangulation $N$ with subdividing map $f:N\to\RR$}
	\KwResult{A closed, orientable 3--dimensional cell complex $M$ that is a subdivision of $N$}
	\Begin{
		$G = \{f(e)\;|\;e\in N^1\}\cup\{e_\varepsilon^\pm\;|\;e\in N^1\}$\;
		\ForEach{tetrahedron $\sigma$ of $N^3$}{
			We construct a 3--dimensional cell complex $\tau$ that is the subdivision of $\sigma$ prescribed by $f$\;
			$\tau = \sigma$\;
			\ForEach{segment $g$ in $G$ such that $g\cap f(\sigma)\neq\emptyset$ and $g\nsubseteq \pd f(\sigma)$}{
				$\delta = $ the intersection of $\sigma$ with $f\inv g$\;
				\ForEach{intersection $d_0 \in\delta\cap\tau^1$ such that $d_0\notin \tau^1$}{
					$d_0 = \delta\cap t_1$ for some 1--cell $t_1$ of $\tau$\;
					Subdivide $t_1$ by $d_0$ into a pair of 1--cells $td_1$ and $dt_1$\;
				}
				\ForEach{intersection $d_1\in\delta\cap\tau^2$}{
					$d_1 = \delta\cap t_2$ for some 2--cell $t_2$ of $\tau$\;
					Subdivide $t_2$ by $d_1$ into a pair of 2--cells $td_2$ and $dt_2$\;
				}
				\ForEach{intersection $d_2\in\delta\cap\tau^3$}{
					$d_2 = \delta\cap t_3$ for some 3--cell $t_3$ of $\tau$\;
					Subdivide $t_3$ by $d_2$ into a pair of 3--cells $td_3$ and $dt_3$\;
				}
			}
		}
		$M = $ the set of 3--complexes $\tau$ constructed above, glued together as prescribed by the gluing maps of $N$ and the identification of the $\tau$ as subdivisions of the tetrahedra of $N$\;
	}
\end{algorithm}


We subdivided $N$ into $M$ in order to identify analogues to the face, edge, and vertex blocks of Chapter \ref{chapter:smooth}.
The analogous blocks are defined exactly as they were in Chapter \ref{chapter:smooth} --- the sleeve segments subdivide the plane into regions homeomorphic to disks, those disks are classified by whether they contain a crossing of $f(N^1)$, intersect $f(N^1)$ but do not contain a crossing, or do not intersect $f(N^1)$ at all.
The preimage of a region that contains a crossing is a \emph{combinatorial vertex block}, the preimage of a region that intersects $f(N^1)$ but does not contain a crossing is an \emph{combinatorial edge block}, and the preimage of a region that is disjoint from $f(N^1)$ is a \emph{combinatorial face block}.
Symmetric names are used for the regions that these blocks project onto --- combinatorial face (resp. edge, vertex) blocks map through $f$ to \emph{combinatorial face (}resp. \emph{edge, vertex) regions}.
Figure \ref{fig:pl-regions} illustrates the regions in the plane that produce such blocks.
These preimages exhaust the cells of $M$, providing a decomposition into subcomplexes.
This decomposition is not a partition --- some cells are assigned to more than one block.
Such an assignment happens precisely when a cell is mapped through $f$ to the shared boundary of different combinatorial regions.

%Note: we 'need' to include S^1\in\RR to actually form the regions discussed, but that's pretty obvious so we won't address it unless requested.



The Cartesian product of a pair of cell complexes is again a cell complex, so we form the base 4--dimensional cell-complex to which we attach handles as $M\times\Ilit$.
Every cell of $M\times\Ilit$ is homeomorphic to a disk, so we form a 4--manifold triangulation $W$ as a subdivision of $M\times\Ilit$.
This triangulation is obtained by inductively subdividing each cell of $M\times\Ilit$ that is not a simplex.




%Each 3--cell of $M$ maps through $f$ to a connected component of $f(N)\setminus f(N^1)$.
%Let $r$ be a connected component of $f(N)\setminus f(N^1)$.
%Then $f\inv(r)$ is a collection of 2-- and 3--cells in $M$ that form a  
%Their is a correspondence between the 3--cells of $M$ and the connected components of $f(N)\setminus f(N^1)$, and that correspondence 
%
%Decomposition of $\RR$ is done through $f(N_1)$.
%A point in $f(N_1)$ is the image of either a vertex, exactly one edge, or exactly two edges (i.e.\ is an edge crossing), so we refer to these as the \emph{vertices}, \emph{edges}, and \emph{crossings} of the decomposition.
%Because $f(N)\setminus f(N_1)$  is a disjoint collection of simply connected regions, we call the connected components of $f(N)\setminus f(N_1)$ the \emph{faces} of the decomposition.
%
%We construct our subdivision of $N$ using the decomposition component preimages.
%The preimage of a face component defines substructures analogous to face blocks, of edge components to edge blocks, and vertices and crossings to vertex blocks.
%Inside of an individual tetrahedron of $N$, edge and crossing preimages are well-defined, supplying a natural subdivision of $N$ into a cell complex.
%Then, the subdivision of a cell complex into a triangulation is well-defined.
%
%The algorithm presented in this section takes as input a closed, orientable 3--manifold triangulation $N$ and a projection $f:N\to\RR$ and produces as output a closed, orientable 3--dimensional cell complex $M$ that is a subdivision of $N$.
%Furthermore, the 3--cells of $M$ are partitioned into subsets that serve the same purpose as the face blocks of Chapter \ref{chapter:smooth}: attaching regions for 2--handles.
%This 3--cell partitioning also induces a partition on the 2--cells of $M$ into subsets that are either interior to face blocks, or subsets that serve the same purpose as the edge blocks of Chapter ~\ref{chapter:smooth}, i.e.\ buffers between face blocks.
%
%
%
%
%
