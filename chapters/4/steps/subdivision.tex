\section{Induce subdivision}
\label{section:pl-subdivide}

The goal of subdividing $N$ is to create and identify handle attachment sites analogous to the face, edge, and vertex blocks of Chapter \ref{chapter:smooth}.
We use a similar technique to that found in Chapter \ref{chapter:smooth}, iteratively subdividing the tetrahedra of $N$ by certain preimages of $f$.
Tetrahedron subdivisions are compatible, i.e.\ tetrahedron subdivisions fit together exactly as the undivided tetrahedra do inside of $N$.

Let $\sigma$ be a tetrahedron of $N$.
If $e$ is an edge of $N^1\setminus\sigma^1$ such that $f(e)\cap f(\sigma)$ is nonempty, then $f(e)\cap f(\sigma)$ is a line segment in $f(\sigma)$ disjoint from any vertices or crossings of $f(\sigma)$.
Figure \ref{fig:standard-position-intersection} demonstrates the possible configurations of these line segments, and shows that their preimages inside of $\sigma$ are triangles and quads.
We refer to these preimages as \emph{exterior triangles} and \emph{exterior quads}.
When a pair of these preimages intersect, the intersection is a line segment with endpoints interior to a pair of triangles of $\sigma^2$.

\begin{figure}[h!]
	\caption{
		\textbf{A tetrahedron $\sigma$ in standard position, intersecting edges, and preimage triangles and quads.}
		An intersecting edge separates the vertices of $\sigma$.
		If one is separated from the other three, its preimage is a triangle.
		If the vertices are separated into two pairs of two, the preimage is a quad.
	}
	\label{fig:standard-position-intersection}
\end{figure}

There are two special preimage triangles in $\sigma$ that we refer to as \emph{interior triangles}.
These occur as the preimages of the edges of $\sigma$ that map through $f$ across $f(\sigma)$ as in Figure \ref{fig:standard-position}, and we refer to these edges as \emph{interior edges}.
The three edges of the interior triangle induced by $e$ are $e$ itself along with a pair of edges that bisect the triangles of $\sigma$ not containing $e$ as an edge.
The three vertices of the interior triangle induced by $e$ are the two vertices of $\sigma$ that are the endpoints of $e$ along with a third vertex located in the edge of $\sigma$ opposite $e$.
Figure \ref{fig:standard-position-interior-exterior} shows one interior triangle along with its possible intersections with exterior triangles and quads.
The interior triangles of $\sigma$ always intersect in a line segment with endpoints inside of the interior edges of $\sigma$, shown in Figure \ref{fig:standard-position-interior-interior}

\begin{figure}[h!]
	\begin{subfigure}[t]{0.45\textwidth}
		\caption{One interior triangle is shown, along with its intersection with an exterior triangle and an exterior quad.} \label{fig:standard-position-interior-exterior}
	\end{subfigure}
	\hfill
	\begin{subfigure}[t]{0.45\textwidth}
		\caption{Both interior triangles are shown, along with their intersection.} \label{fig:standard-position-interior-interior}
	\end{subfigure}
	
	\caption{
		\textbf{A tetrahedron $\sigma$ in standard position, interior triangles, and intersections with external triangles and quads.}
		There are two special preimage triangles in $\sigma$, called \emph{interior triangles}, that occur as the preimages of the edges of $\sigma$ that map through $f$ across $f(\sigma)$ as in Figure \ref{fig:standard-position}.
	}
	\label{fig:standard-position-interior}
\end{figure}

The triangles, quads, and intersections discussed define a subdivision of $\sigma$ into a cell complex.
Iteration of this subdivision in Algorithm \ref{alg:subdividing-manifold} across all tetrahedra of $N$ yields a 3--dimensional cell complex $M$.

\begin{algorithm}
	\caption{Subdividing $N$}
	\label{alg:subdividing-manifold}
	\KwData{A closed, orientable 3--manifold triangulation $N$ with subdividing map $f:N\to\RR$}
	\KwResult{A closed, orientable 3--dimensional cell complex $M$}
	\Begin{
		\ForEach{tetrahedron $\sigma$ of $N^3$}{
			\ForEach{edge $e$ of $N^1$}{
				$\delta = $ the intersection of $\sigma$ with $f\inv\big(f(e)\cap f(\sigma)\big)$\;
				\If{$\delta$ is a triangle or quad in $\sigma$}{
					replace $\sigma$ with the 3--dimensional cell complex obtained as $\sigma$ subdivided by $\delta$
				}
			}
		}
	}
\end{algorithm}

We subdivided $M$ into $N$ in order to identify analogues to the face, edge, and vertex blocks of Chapter \ref{chapter:smooth}.
Face block analogues are the preimages of the components of $f(N)\setminus f(N_1)$, edge block analogues are the preimages of the line segments that comprise $f(e)$ for each $e\in N^1$, and vertex block analogues are the preimages of the edge crossings and vertices of $f$.
These preimages exhaust the cells of $M$, providing a decomposition into subcomplexes.

The Cartesian product of a pair of cell complexes is again a cell complex, so we form the base 4--dimensional cell-complex to which we attach handles as $M\times\Ilit$.
Every cell of $M\times\Ilit$ is homeomorphic to a disk, so we form a 4--manifold triangulation $W$ as a subdivision of $M\times\Ilit$.
This triangulation is obtained by inductively subdividing each cell of $M\times\Ilit$ that is not a simplex.

\newpage

%Each 3--cell of $M$ maps through $f$ to a connected component of $f(N)\setminus f(N^1)$.
%Let $r$ be a connected component of $f(N)\setminus f(N^1)$.
%Then $f\inv(r)$ is a collection of 2-- and 3--cells in $M$ that form a  
%Their is a correspondence between the 3--cells of $M$ and the connected components of $f(N)\setminus f(N^1)$, and that correspondence 
%
%Decomposition of $\RR$ is done through $f(N_1)$.
%A point in $f(N_1)$ is the image of either a vertex, exactly one edge, or exactly two edges (i.e.\ is an edge crossing), so we refer to these as the \emph{vertices}, \emph{edges}, and \emph{crossings} of the decomposition.
%Because $f(N)\setminus f(N_1)$  is a disjoint collection of simply connected regions, we call the connected components of $f(N)\setminus f(N_1)$ the \emph{faces} of the decomposition.
%
%We construct our subdivision of $N$ using the decomposition component preimages.
%The preimage of a face component defines substructures analogous to face blocks, of edge components to edge blocks, and vertices and crossings to vertex blocks.
%Inside of an individual tetrahedron of $N$, edge and crossing preimages are well-defined, supplying a natural subdivision of $N$ into a cell complex.
%Then, the subdivision of a cell complex into a triangulation is well-defined.
%
%The algorithm presented in this section takes as input a closed, orientable 3--manifold triangulation $N$ and a projection $f:N\to\RR$ and produces as output a closed, orientable 3--dimensional cell complex $M$ that is a subdivision of $N$.
%Furthermore, the 3--cells of $M$ are partitioned into subsets that serve the same purpose as the face blocks of Chapter \ref{chapter:smooth}: attaching regions for 2--handles.
%This 3--cell partitioning also induces a partition on the 2--cells of $M$ into subsets that are either interior to face blocks, or subsets that serve the same purpose as the edge blocks of Chapter ~\ref{chapter:smooth}, i.e.\ buffers between face blocks.
%
%
%
%
%
