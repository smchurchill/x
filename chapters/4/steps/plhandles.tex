\section{Attach Handles}
\label{section:pl-handle}

At this point in the chapter we have constructed a 4--manifold triangulation $W$ with two boundary components, each equivalent via subdivision to the 3--dimensional cell complex $M$ obtained in Section \ref{section:pl-subdivide}.
Furthermore, the simplices of $M$ is partitioned into subtriangulations analogous to the face, edge, and vertex blocks of Chapter \ref{chapter:smooth}.
Explicit attachment sites are available for 2--handles, so our first step is construction of a 2--handle.

\subsection{2--handles}
\label{section:pl-2-handle}

The algorithm of this subsection takes as input a pair $(T,\Gamma)$ where $T$ is a closed solid torus triangulation and $\Gamma$ is a collection of triangulated disjoint parallel longitudes in the boundary of $T$.
The output is a 4--disk triangulation $H$ satisfying the following:
\begin{enumerate}
	\item $\pd H$ is a triangulated $S^3$ with a genus 1 Heegard splitting over $\pd T$, i.e.\ $\pd H\setminus \interior{T}$ is a solid torus.
	\item For each $\gamma_i$ in $\Gamma$, $\gamma_i$ bounds a triangulated disk in $\pd H$, i.e.\ each $\gamma_i$ is an explicit 0-framing for the 2--handle attachment.
\end{enumerate}
We take $H$ to be a 4--dimensional 2--handle and attach $H$ to $W$ over $T$.
Constructing the 2--handle involves building a solid torus $T'$ that complements $T$, combining $T$ and $T'$ over their shared boundary to build a triangulated 3--sphere, then coning that 3--sphere into the 4--disk $H$.

We find pairs $(T,\Gamma)$ inside of $M_1\subset\pd W$, the subdivision of $N$ induced by the subdividing map $f:N\to\RR$.
Let $B$ be a combinatorial face block of $M_1$.
$B$ is a 3--dimensional subtriangulation of $M_1$ that forms a closed solid torus,
% Prove?
and $B$ projects through $f$ over an $n$--gon for some $n$.
Call the $n$ corners of $f(B)$ by $c_i$, $i=1\dots n$, and take $C=\{c_1\dots c_n\}$.
Then the $f\inv c_i$ are disjoint parallel triangulated longitudes of $B$.
We take $(B, f\inv C)$ as our pair $(T,\Gamma)$ for each combinatorial face block $B$ of $M$.

For a closed solid torus $T$, we find a complementary torus $T'$, i.e.\ $\pd T=\pd T'$ as triangulations.
We begin by investigating the boundary of $T$, setting $T_0'=\pd T$.
Inside of $T_0'$ is the set of disjoint parallel longitudes $\Gamma$ that are required to bound disks inside of $T'$, so for each $\gamma_i$ we attach a disk $D_i$ with $\pd D_i=\gamma_i$ to $T'_0$ over $\gamma_i$.
Call the result of these attachments $T_1'$.
Adjacent longitudes $\gamma_i$ and $\gamma_j$ bound an annulus $A_{ij}$ in $T_1'$, and $D_i \cup A_{ij}\cup D_j$ is a triangulated 2--sphere that we cone into a 3--disk $D_{ij}$.
For each $ij$, we attach the 3--disk $D_{ij}$ to $T_1'$ over $D_i\cup A_{ij}\cup D_j$ to form the closed solid torus $T'$.

By construction $\pd T'=\pd T$, hence the tori are complementary.
%, so $T$ and $T'$ are complementary.
The longitudinal curves of $T$, the $\gamma_i$, bound triangulated disks in $T'$ thus
% the $\gamma_i$
are meridians of $T'$.
Hence, gluing $T$ and $T'$ over their shared boundary forms a 3--sphere.
We then cone that 3--sphere to produce a 4--disk with an explicit 2--handle structure.
This is the basic structure of Algorithm \ref{alg:build-2-handle}, which we iterate over the combinatorial face blocks of $M_1\subset \pd W$.
Finally, we attach these 2--handles to $W$, forming $W'$.

Let $\mathfrak{B}$ be the collection of combinatorial face blocks of $M_1\subset \pd W$.
Iterating Algorithm \ref{alg:build-2-handle} over $\mathfrak{B}$ yields a collection of 4--dimensional 2--handles $\{H_B^2\}_{B\in\mathfrak{B}}$.
The combinatorial face blocks in $\mathfrak{B}$ are disjoint by construction, so we attach our 2--handles in any order.
We form the 4--manifold triangulation $W'$ from $W$ as
\[
	W' = W\cup\{H_B^2\}_{B\in\mathfrak{B}}/\sim,
\]
where $\sim$ is defined by $b\sim \iota(b)$, $\iota$ the identity map $H_B^2\supset B\overset{\iota}{\to} B\subset M_1$.



\begin{algorithm}[h!]
	\caption{Constructing a 2--handle}
	\label{alg:build-2-handle}
	\KwData{A pair $(T,\Gamma)$ where $T$ is a closed solid torus triangulation and $\Gamma$ is a collection of disjoint parallel triangulated longitudes in $\pd T$}
	\KwResult{A 4--disk $H_T^2$ such that $\pd H_T^2$ is a triangulated 3--sphere with genus 1 Heegard splitting over $\pd T$ and such that $\gamma_i\in\Gamma$ bounds a triangulated disk in $\pd H_T^2\setminus \interior{T}$ for each $i$}
	\Begin{
		First, we construct a complementary solid torus $T'$ with $\pd T=\pd T'$\;
		$T_0'=\pd T$\;
		\ForEach{$\gamma_i\in \Gamma$}{
			$D_i=C(\gamma_i)$, the cone of $\gamma_i$\;
		}
		$T_1'=T_0'\cup\{D_i\}/\sim$, where $\sim$ is induced by 
		%		defined by $g\sim \iota (g)$, $\iota$ 
		%		the identity
		%		 map
		$\pd D_i=\gamma_i\overset{\iota}{\to}\gamma_i\subset T_0'$\;
		\ForEach{adjacent pair of longitudes $\gamma_i,\gamma_j$ from $\Gamma$}{
			$A_{ij}=$ the annulus in $T_1'$ bounded by $\gamma_i$ and $\gamma_j$\;
			$S_{ij}=$ the triangulated 2--sphere in $T_1'$ bounded by $D_i\cup A_{ij}\cup D_j$\;
			$D_{ij}=C(S_{ij})$, the cone of $S_{ij}$\;
		}
		$T'=T_1'\cup\{D_{ij}/\sim\}$, where $\sim$ is induced by
		%		 defined by $s\sim\iota(s)$, $\iota$ the identity map
		$\pd D_{ij}=S_{ij}\overset{\iota}{\to}S_{ij}\subset T_1'$\;
		$S_T^3=T\cup T'/\sim$, where $\sim$ is induced by
		%		 defined by $t\sim\iota(t)$, $\iota$ the identity map 
		$\pd T\overset{\iota}{\to}\pd T'$\;
		$H_T^2=C(S_T^3)$, the cone of $S_T^3$\;
	}
\end{algorithm}

%$B$ possesses the structure of a 3--disk if and only if for some edge $e$ of $B$
%The preimages of the corners of $f(B)$ are parallel longitudes of $B$, and we form a triangulated 2--handle containing $B$ by demanding that these longitudes each bound disks in the 2--handle.

%\begin{figure}[h!]
%	\begin{subfigure}[t]{0.45\textwidth}
%		\caption{Face block analogue: torus.} \label{fig:regular-pl-block-torus}
%	\end{subfigure}
%	\hfill
%	\begin{subfigure}[t]{0.45\textwidth}
%		\caption{Face block analogue: disk.} \label{fig:regular-pl-block-disk}
%	\end{subfigure}
%	
%	\caption{
%		\textbf{A face block analogue $B$ of $M$ and its projection through $f$ in the plane.}
%		$f(B)$ is a convex $n$--gon in the plane for some $n$ that is homeomorphic to either a closed solid torus or a 3--disk.
%%		The preimages of the corners of $f(B)$ are parallel longitudes of $B$, and we form a triangulated 2--handle containing $B$ by demanding that these longitudes each bound disks in the 2--handle.
%	}
%	\label{fig:regular-pl-blocks}
%\end{figure}


\subsection{3--handles}

The manifold $W'$ is formed by attaching triangulated 2-handles to $W$ over the combinatorial face block of $M_1$.
The boundary of $W'$ consists of $M_0 \cup M_1'$, where $M_1'=\pd W'\setminus M_0$ and is related to $M_1$ by replacing each combinatorial face block of $M_1$ with its complement.
Similar to what we saw in Chapter \ref{chapter:smooth}, $M_1'$ can be decomposed into subtriangulations, each of which is a primed combinatorial edge or vertex block.

Primed combinatorial edge blocks are the result of filling annular boundary components of combinatorial edge blocks in with cylinders.
Let $E$ be a combinatorial edge block.
Then $E$ is the product of an orientable surface with an interval, i.e. $E=\Sigma\times\Ilit$, where $\Sigma$ is $S^2$ minus some number of disjoint open balls.
The cylinders that are filled in due to 2--handle attachment are precisely the annuli $\pd\Sigma\times\Ilit$, hence the resulting primed combinatorial edge block is a triangulated $S^2\times\Ilit$.

Assigning cylinders from combinatorial face block complements to primed combinatorial edge blocks exhausts the new cells of $M_1'$ introduced by 2--handle attachment.
We thus set the primed combinatorial vertex blocks of $M_1'$ to be exactly the original combinatorial vertex blocks of $M_1$.

We now construct 3--handles to attach over our primed combinatorial edge blocks.
Let $E'$ be such a block.
Then $\pd E'$ is a disjoint pair of triangulated 2--spheres.
We form a 4--disk containing $E'$ in its boundary by a double coning method on $E'$: we first cone the spherical boundary components of $E'$ to form a pair of triangulated 3--disks, glue these 3--disks to $E'$ to form a 3--sphere, then cone the 3--sphere to obtain a 4--disk.
This idea is identical to that found in Section \ref{subsec:smooth-3handles} and is further formalized in Algorithm \ref{alg:build-3-handle}.

\begin{algorithm}[h!]
	\caption{Constructing a 3--handle}
	\label{alg:build-3-handle}
	\KwData{$E'$, a triangulated $S^2\times\Ilit$}
	\KwResult{A 4--disk $H_{E'}^3$ such that $\pd H_{E'}^3$ is a triangulated 3--sphere containing $E'$ as a subtriangulation}
	\Begin{
		First, we construct a pair of 3--disks to attach over $\pd E'$\;
		\ForEach{boundary sphere $S^2_i\in \pd E'$}{
			$C(S^2_i)=$ the cone of $S^2_i$\;
		}
		$S^3_{E'}=E'\cup {C(S^2_i)}_i$
		The result is a 3--sphere, $S^3_{E'}$\;
		Then $H_{E'}^3=C(S^3_{E'})$\;
	}
\end{algorithm}


