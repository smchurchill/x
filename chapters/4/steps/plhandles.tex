\section{Attach Handles}
\label{section:pl-handle}

At this point in the chapter we have constructed a 4--manifold triangulation $W$ with two boundary components, each equivalent to the 3--dimensional cell complex $M$ obtained in Section \ref{section:pl-subdivide}.
Furthermore, the cells of $M$ are partitioned into subcomplexes analogous to the face, edge, and vertex blocks of Chapter \ref{chapter:smooth}.
Explicit attachment sites are available for 2--handles, so our first step is construction of a 2--handle.

\subsection{Attach 2--handles}
\label{section:pl-2-handle}

The algorithm of this subsection takes as input a closed solid torus triangulation $T$ with a collection of triangulated parallel longitudes $\Gamma$ in its boundary (each $\gamma_i$ in $\Gamma$ is an explicit 0-framing for the 2--handle attachment) and produces as output a $D^4$ triangulation whose $S^3$ boundary triangulation has genus 1 Heegard splitting over $\pd T$, and $\gamma_i$ bounds a triangulated disk in $\pd D^4\setminus T$ for each $i$.
Such a $D^4$ triangulation is taken as a 4--dimensional 2--handle and attached to $W$ over $T$.

To produce pairs $(T,\Gamma)$ we first consider $M$, the subdivision of $N$ induced by the subdividing map $f:N\to\RR$.
Let $B$ be a face block analogue of $M$.
$B$ is a 3--dimensional subcomplex of $M$ whose interior projects over a connected component of $\RR\setminus f(N_1)$, as seen in Figure \ref{fig:regular-pl-blocks}.
Furthermore, $f(B)$ is an $n$--gon for some $n$ whose boundary is inside of $f(N_1)$.
There are two possible structures that $B$ can possess.

\begin{lem}
	Let $B$ be a face block analogue in a 3--dimensional cell complex $M$ obtained as the subdivision of a closed, orientable 3--manifold triangulation $N$ induced by the subdividing map $f:N\to\RR$.
	Then $B$ is homeomorphic to either a closed solid torus or a closed 3--disk.	
\end{lem}

\begin{proof}
		
\end{proof}

%$B$ possesses the structure of a 3--disk if and only if for some edge $e$ of $B$
%The preimages of the corners of $f(B)$ are parallel longitudes of $B$, and we form a triangulated 2--handle containing $B$ by demanding that these longitudes each bound disks in the 2--handle.

\begin{figure}[h!]
	\begin{subfigure}[t]{0.45\textwidth}
		\caption{Face block analogue: torus.} \label{fig:regular-pl-block-torus}
	\end{subfigure}
	\hfill
	\begin{subfigure}[t]{0.45\textwidth}
		\caption{Face block analogue: disk.} \label{fig:regular-pl-block-disk}
	\end{subfigure}
	
	\caption{
		\textbf{A face block analogue $B$ of $M$ and its projection through $f$ in the plane.}
		$f(B)$ is a convex $n$--gon in the plane for some $n$ that is homeomorphic to either a closed solid torus or a 3--disk.
%		The preimages of the corners of $f(B)$ are parallel longitudes of $B$, and we form a triangulated 2--handle containing $B$ by demanding that these longitudes each bound disks in the 2--handle.
	}
	\label{fig:regular-pl-blocks}
\end{figure}


\subsection{Attach 3--handles}

We now have a triangulated 4--manifold with boundary components $M_0$ and the result of surgery on $M_1$ induced by 2--handle attachment.
This surgery had the following effect on $M_1$:
\begin{enumerate}
	\item Each solid torus in the partition of $M_1$ induced by the projection from Section ~\ref{section:pl-projection} is associated with a triangulated $D^4$ in Section ~\ref{section:pl-2-handle}.
	\item For each (torus, 2--handle) pair $(T, H^2)$, the boundary of $H^2$ contains $T$ and $T^* = H^2\setminus \interior{T}$ is another solid torus whose boundary triangulation is identical to that of $T$.
	\item Thus the effect of surgery on $M_1$ is of replacing each $T$ with $T^*$ over their shared boundary.
\end{enumerate}

