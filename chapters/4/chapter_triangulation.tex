% !TeX root = ../../smc-thesis.tex
\chapter[Triangulated]{Algorithm for constructing a triangulated 4--manifold with prescribed 3--manifold boundary}
\label{chapter:triangulation}

The steps to construct a triangulated 4--manifold with prescribed closed, orientable 3--manifold boundary broadly follow the steps to construct a 4--manifold with prescribed smooth, orientable 3--manifold boundary.
Let $N$ be a closed, orientable 3--manifold triangulation.
Then the steps of construction are:
\begin{enumerate}
	\item Define a projection $f:N\to\RR$.
	
	\item Induce a subdivision of $N$ from $f$.  The result is a 3--dimensional cell complex $M$ that is equivalent to $N$.

	\item Let $W=M\times\Ilit$ be a 4--dimensional cell complex with boundary components $M_0 = M\times\{0\}$ and $M_1 = M\times\{1\}$.
	
	\item Attach 4--dimensional 2--handles to $W$ over its $M_1$ boundary as prescribed by the subdivision of $M$ from $f$.  Call the result $W'$ and call the boundary of $W'$ different from $M_0$ by $M_1'$.
	
	\item Attach 4--dimensional 3--handles to $W'$ over $M_1'$ as prescribed by the subdivision induced by $f$ and the surgery induced by 2--handle attachment.  Call the result $W''$.
	
	\item The boundary of $W''$ consists of $M_0$ and a collection of copies of $S^3$ that we now cone off.  The result is a 4--dimensional cell complex whose boundary is exactly $M_0$.
\end{enumerate}

Each of these steps is made algorithmic, and these algorithms are chained in series to form a single algorithm.
This algorithm has input a closed, orientable 3--manifold triangulation $N$ and output a 4--manifold triangulation $W$ whose sole boundary component is a triangulated 3--manifold that is equivalent to $N$ in the sense of triangulations.
In this case, we find that $\pd W$ is a subdivision of $N$, and this subdivision is the subdivision induced by the projection $f$ in Step 1.

It is necessary that the input 3--manifold triangulation is \emph{edge-distinct}, i.e.\ if $u,v$ are vertices of $N$ then $\{u,v\}$ is the boundary of at most one edge.
If this condition is not satisfied by a given $N$, then it is satisfied by the barycentric subdivision of $N$.
We assume that $N$ is edge-distinct for the remainder of the chapter.

Throughout this chapter $N$ refers to the initial input closed, orientable 3--manifold triangulation, $f$ refers to the subdividing map defined in Section \ref{section:pl-projection}, $M$ is the subdivision of $N$ induced by $f$, $W$ is the 4--manifold $M\times\Ilit$, $W'$ is the result of attaching 2--handles to $W$, and $W''$ is the result of attaching 3--handles to $W'$.

\section{Define projection}
\label{section:pl-projection}

The projection's utility is in defining a subdivision of the initial closed, orientable 3--manifold triangulation $N$ such that attaching regions for triangulated 2-- and 3--handles can be found.
This is done before forming the base 4--manifold so that the subdivided triangulation is used in the algorithm that provides $W$.

Our subdivision is obtained by imposing four conditions on $f:N\to\RR$:
\begin{enumerate}
	\item $f$ maps vertices to the circle, i.e.\ for each vertex $v\in N^0$, $f(v)$ lies on the unit circle in $\RR$.
	
	\item The images of vertices are distinct, i.e.\ for every pair of vertices $u,v\in N^0$, $f(u)\neq f(v)$.
	
	\item $f$ is linear on each simplex of $N$ and piecewise-linear on $N$, i.e.\ if $x\in\sigma$ is a point in the simplex $\sigma$ with vertices $v_i$, then $x=\sum_i a_i v_i$ with $\sum_i a_i = 1$ and $f(x) = \sum_i a_i f(v_i)$.
	
	\item Edge intersections are distinct, i.e.\ for every triple of edges $e_1, e_2, e_3\in N^1$ that share no vertices, $f(e_1)\cap f(e_2)\neq f(e_2)\cap f(e_3)$.
\end{enumerate}

Conditions 1 and 2 ensure that every simplex of $N$ is mapped to the plane in standard position (i.e.\ the images of the vertices in the plane form a convex set).
This, along with conditions 3 and 4, allows us to use concepts and language from normal surface theory to describe the subdivision of $N$ in the next section.
We call these four conditions the \emph{subdivision conditions} on $f$.

All conditions are satisfied by fixing an odd integer $k$ greater than or equal to the number of vertices in $N$, injecting the vertices of $N$ to the $k\nth$ complex roots of unity in the plane, then extending linearly over the skeletons of $N$.
The first three conditions are clearly satisfied by this procedure, and the last is satisfied by applying the results in \cite{PoonRub98}.

The algorithm presented in this section takes as input the triangulated 3--manifold $N$ and produces a projection $f:N\to\RR$ satisfying the subdivision conditions.

\section{Induce subdivision}

The goal of subdividing $N$ is to create and identify analogues to the face, edge, and vertex blocks of Chapter \ref{chapter:smooth} where we may iteratively attach 2--, 3--, then 4-- handles.
We use a similar technique to that found in Chapter \ref{chapter:smooth}, first decomposing $\RR$ with the projection, then examining preimages to define our subdivision.

Decomposition of $\RR$ is done through $f(N_1)$.
A point in $f(N_1)$ is the image of either a vertex, exactly one edge, or exactly two edges (i.e.\ is an edge crossing), so we refer to these as the \emph{vertices}, \emph{edges}, and \emph{crossings} of the decomposition.
Because $f(N)\setminus f(N_1)$  is a disjoint collection of simply connected regions, we call the connected components of $f(N)\setminus f(N_1)$ the \emph{faces} of the decomposition.

We construct our subdivision of $N$ using the decomposition component preimages.
The preimage of a face component defines substructures analogous to face blocks, of edge components to edge blocks, and vertices and crossings to vertex blocks.
Inside of an individual tetrahedron of $N$, edge and crossing preimages are well-defined, supplying a natural subdivision of $N$ into a cell complex.
Then, the subdivision of a cell complex into a triangulation is well-defined.

The algorithm presented in this section takes as input a closed, orientable 3--manifold triangulation $N$ and a projection $f:N\to\RR$ and produces as output a closed, orientable 3--manifold triangulation $M$ that is a subdivision of $N$.
Furthermore,
\begin{enumerate}
	\item the 3--cells of $M$ are partitioned into subsets that serve the same purpose as the face blocks of Chapter ~\ref{chapter:smooth}: attaching regions for 2--handles,
	\item the 2--cells of $M$ are partitioned into subsets that are either interior to face blocks, or subsets that serve the same purpose as the edge blocks of Chapter ~\ref{chapter:smooth}, i.e.\ buffers between face blocks.	
\end{enumerate}


\section{Form base 4--manifold}

\section{Attach 2--handles}
\label{section:pl-2-handle}

At this point in our procedure we have a 4--manifold $W$ with triangulated boundary components $M_0$ and $M_1$.
We aim to attach handles to $W$ over the boundary of $W$ away from $M_0$ until the boundary of $W$ is exactly $M_0$.
The first step is to attach 2--handles to $W$ over the closed solid torus triangulations that partition $M_1$.

The algorithm of this section takes as input a closed solid torus triangulation $T$ with a collection of triangulated parallel longitudes $\Gamma$ in its boundary (each $\gamma_i$ in $\Gamma$ is an explicit 0-framing for the 2--handle attachment) and produces as output a $D^4$ triangulation whose $S^3$ boundary triangulation has genus 1 Heegard splitting over $\pd T$, and $\gamma_i$ bounds a triangulated disk in $\pd D^4\setminus T$ for each $i$.
Such a $D^4$ triangulation is taken as a 4--dimensional 2--handle and attached to $W$ over $T$.
We then attach such a 2--handle over each closed solid torus in our partition of $M_0$.

\section{Attach 3--handles}

We now have a triangulated 4--manifold with boundary components $M_0$ and the result of surgery on $M_1$ induced by 2--handle attachment.
This surgery had the following effect on $M_1$:
\begin{enumerate}
	\item Each solid torus in the partition of $M_1$ induced by the projection from Section ~\ref{section:pl-projection} is associated with a triangulated $D^4$ in Section ~\ref{section:pl-2-handle}.
	\item For each (torus, 2--handle) pair $(T, H^2)$, the boundary of $H^2$ contains $T$ and $T^* = H^2\setminus \interior{T}$ is another solid torus whose boundary triangulation is identical to that of $T$.
	\item Thus the effect of surgery on $M_1$ is of replacing each $T$ with $T^*$ over their shared boundary.
\end{enumerate}



%\section{Cone away final boundary components}


