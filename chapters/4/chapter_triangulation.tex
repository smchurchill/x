% !TeX root = ../../smc-thesis.tex
\chapter[Triangulated]{Algorithm for constructing a triangulated 4--manifold with prescribed 3--manifold boundary}
\label{chapter:triangulation}

The steps to construct a triangulated 4--manifold with prescribed closed, orientable 3--manifold boundary broadly follow the steps to construct a 4--manifold with prescribed smooth, orientable 3--manifold boundary.
Let $N$ be a closed, orientable 3--manifold triangulation.
Then the steps of construction are:
\begin{enumerate}
	\item Define a projection $f:N\to\RR$.
	
	\item Induce a subdivision of $N$ from $f$.  The result is a 3--dimensional cell complex $M$ that is equivalent to $N$.

	\item Let $W=M\times\Ilit$ be a 4--dimensional cell complex with boundary components $M_0 = M\times\{0\}$ and $M_1 = M\times\{1\}$.
	
	\item Attach 4--dimensional 2--handles to $W$ over its $M_1$ boundary as prescribed by the subdivision of $M$ from $f$.  Call the result $W'$ and call the boundary of $W'$ different from $M_0$ by $M_1'$.
	
	\item Attach 4--dimensional 3--handles to $W'$ over $M_1'$ as prescribed by the subdivision induced by $f$ and the surgery induced by 2--handle attachment.  Call the result $W''$.
	
	\item The boundary of $W''$ consists of $M_0$ and a collection of copies of $S^3$ that we now cone off.  The result is a 4--dimensional cell complex whose boundary is exactly $M_0$.
\end{enumerate}

Each of these steps is made algorithmic, and these algorithms are chained in series to form a single algorithm.
This algorithm has input a closed, orientable 3--manifold triangulation $N$ and output a 4--manifold triangulation $W$ whose sole boundary component is a triangulated 3--manifold that is equivalent to $N$ in the sense of triangulations.
In this case, we find that $\pd W$ is a subdivision of $N$, and this subdivision is the subdivision induced by the projection $f$ in Step 1.

It is necessary that the input 3--manifold triangulation is \emph{edge-distinct}, i.e.\ if $u,v$ are vertices of $N$ then $\{u,v\}$ is the boundary of at most one edge.
If this condition is not satisfied by a given $N$, then it is satisfied by the barycentric subdivision of $N$.
We assume that $N$ is edge-distinct for the remainder of the chapter.

Throughout this chapter $N$ refers to the initial input closed, orientable 3--manifold triangulation, $f$ refers to the subdividing map defined in Section \ref{section:pl-projection}, $M$ is the subdivision of $N$ induced by $f$, $W$ is the 4--manifold $M\times\Ilit$, $W'$ is the result of attaching 2--handles to $W$, and $W''$ is the result of attaching 3--handles to $W'$.

\section{Define projection}
\label{section:pl-projection}

The projection's utility is in defining a subdivision of the initial closed, orientable 3--manifold triangulation $N$ such that attaching regions for triangulated 2-- and 3--handles can be found.
This is done before forming the base 4--manifold so that the subdivided triangulation is used in the algorithm that provides $W$.

Our subdivision is obtained by imposing four conditions on $f:N\to\RR$:
\begin{enumerate}
	\item $f$ maps vertices to the circle, i.e.\ for each vertex $v\in N^0$, $f(v)$ lies on the unit circle in $\RR$.
	
	\item The images of vertices are distinct, i.e.\ for every pair of vertices $u,v\in N^0$, $f(u)\neq f(v)$.
	
	\item $f$ is linear on each simplex of $N$ and piecewise-linear on $N$, i.e.\ if $x\in\sigma$ is a point in the simplex $\sigma$ with vertices $v_i$, then $x=\sum_i a_i v_i$ with $\sum_i a_i = 1$ and $f(x) = \sum_i a_i f(v_i)$.
	
	\item Edge intersections are distinct, i.e.\ for every triple of edges $e_1, e_2, e_3\in N^1$ that share no vertices, $f(e_1)\cap f(e_2)\neq f(e_2)\cap f(e_3)$.
\end{enumerate}

Conditions 1 and 2 ensure that every simplex of $N$ is mapped to the plane in standard position (i.e.\ the images of the vertices in the plane form a convex set).
This, along with conditions 3 and 4, allows us to use concepts and language from normal surface theory to describe the subdivision of $N$ in the next section.
We call these four conditions the \emph{subdivision conditions} on $f$.

All conditions are satisfied by fixing an odd integer $k$ greater than or equal to the number of vertices in $N$, injecting the vertices of $N$ to the $k\nth$ complex roots of unity in the plane, then extending linearly over the skeletons of $N$.
The first three conditions are clearly satisfied by this procedure, and the last is satisfied by applying the results in \cite{PoonRub98}.

The algorithm presented in this section takes as input the triangulated 3--manifold $N$ and produces a projection $f:N\to\RR$ satisfying the subdivision conditions.

\section{Induce subdivision}

The goal of subdividing $N$ is to create and identify analogues to the face, edge, and vertex blocks of Chapter \ref{chapter:smooth} where we may iteratively attach 2--, 3--, then 4-- handles.
We use a similar technique to that found in Chapter \ref{chapter:smooth}, first decomposing $\RR$ with the projection, then examining preimages to define our subdivision.

Decomposition of $\RR$ is done through $f(N_1)$.
A point in $f(N_1)$ is the image of either a vertex, exactly one edge, or exactly two edges (i.e.\ is an edge crossing), so we refer to these as the \emph{vertices}, \emph{edges}, and \emph{crossings} of the decomposition.
Because $f(N)\setminus f(N_1)$  is a disjoint collection of simply connected regions, we call the connected components of $f(N)\setminus f(N_1)$ the \emph{faces} of the decomposition.

We construct our subdivision of $N$ using the decomposition component preimages.
The preimage of a face component defines substructures analogous to face blocks, of edge components to edge blocks, and vertices and crossings to vertex blocks.
Inside of an individual tetrahedron of $N$, edge and crossing preimages are well-defined, supplying a natural subdivision of $N$ into a cell complex.
Then, the subdivision of a cell complex into a triangulation is well-defined.

The algorithm presented in this section takes as input a closed, orientable 3--manifold triangulation $N$ and a projection $f:N\to\RR$ and produces as output a closed, orientable 3--manifold triangulation $M$ that is a subdivision of $N$.
Furthermore,
\begin{enumerate}
	\item the 3--cells of $M$ are partitioned into subsets that serve the same purpose as the face blocks of Chapter ~\ref{chapter:smooth}: attaching regions for 2--handles,
	\item the 2--cells of $M$ are partitioned into subsets that are either interior to face blocks, or subsets that serve the same purpose as the edge blocks of Chapter ~\ref{chapter:smooth}, i.e.\ buffers between face blocks.	
\end{enumerate}


%\section{Form base 4--manifold}

\section{Attach Handles}
\label{section:pl-handle}

At this point in the chapter we have constructed a 4--manifold triangulation $W$ with two boundary components, each equivalent via subdivision to the 3--dimensional cell complex $M$ obtained in Section \ref{section:pl-subdivide}.
Furthermore, the simplices of $M$ are partitioned into subtriangulations analogous to the face, edge, and vertex blocks of Chapter \ref{chapter:smooth}.
Explicit attachment sites are available for (4,2)--handles, so our first step is construction of a (4,2)--handle.

\subsection{2--handles}
\label{section:pl-2-handle}

The algorithm of this subsection takes as input a pair $(T,\Gamma)$ where $T$ is a closed solid torus triangulation and $\Gamma$ is a collection disjoint parallel longitudes in the boundary of $T$, given as subtriangulations of $\pd T$.
The output is a 4--disk triangulation $H$ satisfying the following:
\begin{enumerate}
	\item $\pd H$ is a triangulated $S^3$ with a genus 1 Heegard splitting over $\pd T$, i.e.\ $T$ is a subtriangulation of $\pd H$ such that $\pd H\setminus \interior{T}$ is a solid torus.
	\item For each $\gamma_i$ in $\Gamma$, $\gamma_i$ bounds a triangulated disk in $\pd H$, i.e.\ each $\gamma_i$ is an explicit 0-framing for the (4,2)--handle attachment.
\end{enumerate}
We take $H$ to be a (4,2)--handle and attach $H$ to $W$ over $T$.
Constructing the handle involves building a solid torus $T'$ that complements $T$, combining $T$ and $T'$ over their shared boundary to build a triangulated 3--sphere, then coning that 3--sphere into the 4--disk $H$.

We find pairs $(T,\Gamma)$ inside $M_1\subset\pd W$, the subdivision of $N$ induced by the subdividing map $f:N\to\RR$.
Let $B$ be a combinatorial face block of $M_1$.
$B$ is a 3--dimensional subtriangulation of $M_1$ that forms a closed solid torus,
% Prove?
and $B$ projects through $f$ over an $n$--gon for some $n$.
Call the $n$ corners of $f(B)$ by $c_i$, $i=1\dots n$, and take $C=\{c_1\dots c_n\}$.
Then the $f\inv c_i$ are disjoint parallel triangulated longitudes of $B$.
We take $(B, f\inv C)$ as our pair $(T,\Gamma)$ for each combinatorial face block $B$ of $M$.

For a closed solid torus $T$, we find a complementary torus $T'$, i.e.\ $\pd T=\pd T'$ as triangulations.
We begin by investigating the boundary of $T$, setting $T_0'=\pd T$.
Inside $T_0'$ is the set of disjoint parallel longitudes $\Gamma$ that are required to bound disks inside $T'$, so for each $\gamma_i$ we attach a disk $D_i$ with $\pd D_i=\gamma_i$ to $T'_0$ over $\gamma_i$.
Call the result of these attachments $T_1'$.
Adjacent longitudes $\gamma_i$ and $\gamma_j$ bound an annulus $A_{ij}$ in $T_1'$, and $D_i \cup A_{ij}\cup D_j$ is a triangulated 2--sphere that we cone into a 3--disk $D_{ij}$.
For each $ij$, we attach the 3--disk $D_{ij}$ to $T_1'$ over $D_i\cup A_{ij}\cup D_j$ to form the closed solid torus $T'$.

By construction $\pd T'=\pd T$, hence the tori are complementary.
%, so $T$ and $T'$ are complementary.
The longitudinal curves of $T$, the $\gamma_i$, bound triangulated disks in $T'$ thus
% the $\gamma_i$
are meridians of $T'$.
Hence, gluing $T$ and $T'$ over their shared boundary forms a 3--sphere.
We then cone that 3--sphere to produce a 4--disk with an explicit (4,2)--handle structure.
This is the basic structure of Algorithm \ref{alg:build-2-handle}, which we iterate over the combinatorial face blocks of $M_1\subset \pd W$.
Finally, we attach these handles to $W$, forming $W'$.

Let $\mathfrak{B}$ be the collection of combinatorial face blocks of $M_1\subset \pd W$.
Iterating Algorithm \ref{alg:build-2-handle} over $\mathfrak{B}$ yields a collection of (4,2)--handles $\{H_B^2\}_{B\in\mathfrak{B}}$.
The combinatorial face blocks in $\mathfrak{B}$ are disjoint by construction, so we attach our handles in any order.
We form the 4--manifold triangulation $W'$ from $W$ as
\[
	W' = W\cup\{H_B^2\}_{B\in\mathfrak{B}}/\sim,
\]
where $\sim$ is defined by $b\sim \iota(b)$, $\iota$ the identity map $H_B^2\supset B\overset{\iota}{\to} B\subset M_1$.



\begin{algorithm}[h!]
	\caption{Constructing a (4,2)--handle}
	\label{alg:build-2-handle}
	\KwData{A pair $(T,\Gamma)$ where $T$ is a closed solid torus triangulation and $\Gamma$ is a collection of disjoint parallel triangulated longitudes in $\pd T$}
	\KwResult{A 4--disk triangulation $H_T^2$ such that $\pd H_T^2$ is a triangulated 3--sphere with genus 1 Heegard splitting over $\pd T$ and such that $\gamma_i\in\Gamma$ bounds a triangulated disk in $\pd H_T^2\setminus \interior{T}$ for each $i$}
	\Begin{
		First, we construct a complementary solid torus $T'$ with $\pd T=\pd T'$\;
		$T_0'=\pd T$\;
		\ForEach{$\gamma_i\in \Gamma$}{
			$D^2_i=C(\gamma_i)$, the cone of $\gamma_i$\;
		}
		$T_1'=T_0'\cup\{D^2_i\}/\sim$, where $\sim$ is induced by 
		%		defined by $g\sim \iota (g)$, $\iota$ 
		%		the identity
		%		 map
		$\pd D^2_i=\gamma_i\overset{\iota}{\to}\gamma_i\subset T_0'$\;
		\ForEach{adjacent pair of longitudes $\gamma_i,\gamma_j$ from $\Gamma$}{
			$A_{ij}=$ the annulus in $T_1'$ bounded by $\gamma_i$ and $\gamma_j$\;
			$S^2_{ij}=$ the triangulated 2--sphere in $T_1'$ that is precisely $D^2_i\cup A_{ij}\cup D^2_j$\;
			$D^3_{ij}=C(S^2_{ij})$, the cone of $S^2_{ij}$\;
		}
		$T'=T_1'\cup\{D^3_{ij}/\sim\}$, where $\sim$ is induced by
		%		 defined by $s\sim\iota(s)$, $\iota$ the identity map
		$\pd D^3_{ij}=S^2_{ij}\overset{\iota}{\to}S^2_{ij}\subset T_1'$\;
		$S_T^3=T\cup T'/\sim$, where $\sim$ is induced by
		%		 defined by $t\sim\iota(t)$, $\iota$ the identity map 
		$\pd T\overset{\iota}{\to}\pd T'$\;
		$H_T^2=C(S_T^3)$, the cone of $S_T^3$\;
	}
\end{algorithm}

%$B$ possesses the structure of a 3--disk if and only if for some edge $e$ of $B$
%The preimages of the corners of $f(B)$ are parallel longitudes of $B$, and we form a triangulated 2--handle containing $B$ by demanding that these longitudes each bound disks in the 2--handle.

%\begin{figure}[h!]
%	\begin{subfigure}[t]{0.45\textwidth}
%		\caption{Face block analogue: torus.} \label{fig:regular-pl-block-torus}
%	\end{subfigure}
%	\hfill
%	\begin{subfigure}[t]{0.45\textwidth}
%		\caption{Face block analogue: disk.} \label{fig:regular-pl-block-disk}
%	\end{subfigure}
%	
%	\caption{
%		\textbf{A face block analogue $B$ of $M$ and its projection through $f$ in the plane.}
%		$f(B)$ is a convex $n$--gon in the plane for some $n$ that is homeomorphic to either a closed solid torus or a 3--disk.
%%		The preimages of the corners of $f(B)$ are parallel longitudes of $B$, and we form a triangulated 2--handle containing $B$ by demanding that these longitudes each bound disks in the 2--handle.
%	}
%	\label{fig:regular-pl-blocks}
%\end{figure}

The manifold $W'$ is formed by attaching triangulated (4,2)--handles to $W$ over the combinatorial face blocks of $M_1$.
The boundary of $W'$ consists of $M_0 \cup M_1'$, where $M_1'=\pd W'\setminus M_0$ and $M_1'$ is related to $M_1$ via triangulated (3,2)--handle attachments to $M_1$ induced by the triangulated (4,2)--handle attachments to $W$.
We decompose $M_1'$ into primed combinatorial edge or vertex block subtriangulations in a manner similar to that found in Chapter \ref{chapter:smooth}.

Primed combinatorial edge blocks are the result of attaching triangulated (3,2)--handles over annular boundary components of combinatorial edge blocks.
These (3,2)--handles are the triangulated 3-disks $D^3_{ij}$ formed on Line 11 of Algorithm \ref{alg:build-2-handle}, and their attaching regions are the annuli of combinatorial edge blocks that project through $f$ over edge-face region boundaries.

Let $E$ be a combinatorial edge block.
Then $E$ is the product of an orientable surface with an interval, i.e. $E=\Sigma\times\Ilit$, where $\Sigma$ is $S^2$ minus some number of disjoint open balls.
The annuli $\pd \Sigma\times\Ilit$ map through $f$ to edge-face region boundaries, hence these are the attaching sites of (3,2)--handles.
Primed combinatorial edge blocks are constructed by performing all such (3,2)--handle attachments, thus each primed combinatorial edge block is a triangulated $S^2\times\Ilit$.

Differing slightly from Chapter \ref{chapter:smooth}, this exhausts the new cells of $M_1'$ introduced by 2--handle attachment.
This is due to the different shape of combinatorial vertex regions as compared to the vertex regions of Chapter \ref{chapter:smooth}.
In Chapter \ref{chapter:smooth} vertex regions were octagonal and shared edge boundaries with face regions.
Here, combinatorial vertex regions are quadrilateral and share point boundaries with combinatorial face regions.
This point boundary is also shared by a pair of combinatorial edge regions.

Let $x$ be a vertex-face-edge-edge boundary.
Then $f\inv(x)=\{\gamma^i\}$ is a collection of triangulated circles, and each $\gamma^i$ is a longitude of a toroidal combinatorial face block in $M_1$ that projects over the face region whose boundary contains $x$.
Thus, every $\gamma^i$ has been filled by a 2--disc constructed on Line 5 of Algorithm \ref{alg:build-2-handle}.
Each of these new disks has been assigned to two primed combinatorial edge blocks, but we will also assign them to the primed vertex block that contains $\gamma^i$.

This means that each primed combinatorial vertex block is now equivalent to a triangulated (3,2)--handlebody.
Let $V$ be a combinatorial vertex block and $V'$ be $V$ primed.
Considering each new disc in $V'$ to be a flattened (3,2)--handle, we conclude that $V'$ is a (3,2)--handlebody using the argument found in Theorem \ref{thm:primed-block-structure}.
The attaching spheres of (3,2)--handles transversely intersect the belt spheres of the (3,1)--handles at most once, each belt sphere intersects at least one attaching sphere and vice versa, and the boundary of $V'$ is a disjoint collection of 2--spheres, thus $V'$ is a (3,2)--handlebody.

\subsection{3--handles}

We now construct (4,3)--handles to attach over our primed combinatorial edge blocks.
Let $E'$ be such a block.
Then $\pd E'$ is a disjoint pair of triangulated 2--spheres.
We form a 4--disk containing $E'$ in its boundary by a double coning method on $E'$: we first cone the spherical boundary components of $E'$ to form a pair of triangulated 3--disks, glue these 3--disks to $E'$ to form a 3--sphere, then cone the 3--sphere to obtain a 4--disk.
This idea is identical to that found in Section \ref{subsec:smooth-3handles} and is formalized in Algorithm \ref{alg:build-3-handle}.
Finally, we attach (4,3)--handles built using Algorithm \ref{alg:build-3-handle} to $W'$ over primed combinatorial edge blocks.

Let $\mathfrak{E}'$ be the collection of primed combinatorial edge blocks of $M_1'\subset \pd W'$.
Iterating Algorithm \ref{alg:build-3-handle} over $\mathfrak{E}'$ yields a collection of (4,3)--handles $\{H_{E'}^3\}_{E'\in\mathfrak{E}'}$.
The primed combinatorial edge blocks in $\mathfrak{E}'$ intersect only in discs that are also shared by primed combinatorial vertex blocks and these disks are invariant under the (4,3)--handle attachment so we attach our handles in any order.
We form the 4--manifold triangulation $W''$ from $W'$ as
\[
W'' = W'\cup\{H_{E'}^3\}_{E'\in\mathfrak{E}'}/\sim,
\]
where $\sim$ is defined by $e\sim \iota(e)$, $\iota$ the identity map $H_{E'}^3\supset E'\overset{\iota}{\to} E'\subset M_1'$.

\begin{algorithm}[h!]
	\caption{Constructing a (4,3)--handle}
	\label{alg:build-3-handle}
	\KwData{$E'$, a triangulated $S^2\times\Ilit$}
	\KwResult{A 4--disk triangulation $H_{E'}^3$ such that $\pd H_{E'}^3$ is a triangulated 3--sphere containing $E'$ as a subtriangulation}
	\Begin{
		First, we attach 3--disks to $E'$ over its two 2-sphere boundary components\;
		\ForEach{boundary sphere $S^2_i\in \pd E'$}{
			$C(S^2_i)=$ the cone of $S^2_i$\;
			Attach $C(S^2_i)$ to $\pd E'$ over $S^2_i\in \pd E'$\;
		}
		The result of the above attachments is the 3--sphere $S^3_{E'}$\;
		Then $H_{E'}^3=C(S^3_{E'})$\;
	}
\end{algorithm}

The manifold $W''$ is formed by attaching triangulated (4,3)--handles to $W'$ over the primed combinatorial edge blocks of $M_1'$.
The boundary of $W''$ consists of $M_0 \cup M_1''$, where $M_1''=\pd W''\setminus M_0$ and $M_1''$ is related to $M_1'$ via triangulated (3,3)--handle attachments to $M_1'$ induced by the triangulated (4,3)--handle attachments to $W'$.
These (3,3)--handle attachments turn primed combinatorial vertex blocks into double primed combinatorial vertex blocks.
A primed combinatorial vertex block is a (3,2)--handlebody, so a  double primed combinatorial vertex block is a (3,2)--handlebody with each $S^2$ boundary component filled in with a (3,3)--handle, thus each double primed combinatorial vertex block is a triangulated $S^3$.

\subsection{4--handles}

We now construct (4,4)--handles to attach over our double primed combinatorial vertex blocks.
Let $V''$ be such a block.
Then $V''$ is a triangulated $S^3$, so a (4,4)--handle structure that can be attached over $V''$ is constructed by coning $V''$.
This is made explicit in Algorithm \ref{alg:build-4-handle}, which serves double duty as a general algorithm for coning spheres into discs.
These handles are attached to $W''$ to form a triangulated 4--manifold whose only boundary component is $M_0$.

Let $\mathfrak{V}''$ be the collection of double primed combinatorial vertex blocks of $M_1''\subset \pd W''$.
Coning each element of $\mathfrak{V}''$ yields a collection of (4,4)--handles $\{H_{V''}^4\}_{V''\in\mathfrak{V}''}$.
The double primed combinatorial vertex blocks in $\mathfrak{V}''$ are disjoint, so we attach our handles in any order.
We form the 4--manifold triangulation $W'''$ from $W''$ as
\[
W''' = W''\cup\{H_{V''}^4\}_{V''\in\mathfrak{V}''}/\sim,
\]
where $\sim$ is defined by $v\sim \iota(v)$, $\iota$ the identity map $H_{V''}^4\supset V''\overset{\iota}{\to} V''\subset M_1''$.

\begin{algorithm}[h!]
	\caption{Constructing a (4,4)--handle; equivalently, coning a sphere}
	\label{alg:build-4-handle}
	\KwData{$V''$, a triangulated $S^3$}
	\KwResult{A 4--disk triangulation $H_{V''}^4$ such that $\pd H_{V''}^4=V''$}
	\Begin{
		$H_{V''}^4= V''$\;
		Add a 0--cell $x$ to $H_{V''}^4$\;
%		\ForEach{0--cell $y$ of $V''$}{
%			Attach a 1--cell $\{x,y\}$ to $H$\;
%%			Track $\{x,y\}$ as a `new' 1--cell\;
%		}
		\For{$i=0\dots 3$}{
			\ForEach{$i$--cell $y$ of $V''\subset H$}{
				$Z(y)=\{z\in H_{V''}^4\;|\; z\notin V'' \textrm{ and }\pd z\cap \pd y \neq\emptyset\}$\; 
				Attach an $(i+1)$--cell to $H_{V''}^4$ over the triangulated $i$--sphere in $H_{V''}^4$ that is precisely $y\cup Z(y)\cup x$\;
			}
		}
	}
\end{algorithm}

The triangulated 4--manifold $W'''$ has only one boundary component: $M_0$.
By construction, $M_0$ is equivalent to the original input 3--manifold triangulation $N$ by subdivision of $N$, hence $W'''$ is a triangulated 4--manifold with boundary $N$, as desired.
Each algorithm in the construction is now chained together in Algorithm \ref{alg:chain}.

\begin{algorithm}[h!]
	\caption{Full construction of a triangulated 4--manifold with prescribed 3--manifold boundary}
	\label{alg:chain}
	\KwData{$N$, a 3--manifold triangulation}
	\KwResult{A 4--manifold triangulation whose boundary is equivalent to $N$ through subdivision}
	\Begin{
		Obtain a subdiving map $f$ for $N$ from Algorithm \ref{alg:subdividing-map}\;
		Construct $M$, a 3--dimensional cell complex equivalent to $N$ from Algorithm \ref{alg:subdividing-manifold} applied to $(N,f)$\;
		Construct $W$, the triangulation obtained by inductively subdividing the cells of $M\times\Ilit$ that are not simplices\;
		$\mathfrak{B}=$ the collection of combinatorial face blocks of $M_1\subset\pd W$\;
		\ForEach{$B\in\mathfrak{B}$}{
			$H_B^2=$ the (4,2)--handle produced by Algorithm \ref{alg:build-2-handle} applied to $B$\;
		}
		$W' = $ the 4--manifold triangulation obtained by attaching $H_B^2$ to $W$ over $B$ for each $B\in\mathfrak{B}$\;
		$M_1' = \pd W'\setminus M_0$\;
		Apportion the cells of $\{H_B^2\}_{B\in\mathfrak{B}}$ to the combinatorial edge and vertex blocks of $W'$ to form primed combinatorial edge and vertex blocks\;
		$\mathfrak{E}'=$ the collection of primed combinatorial edge blocks of $M_1'\subset\pd W'$\;
		\ForEach{$E'\in\mathfrak{E}'$}{
			$H_{E'}^3=$ the (4,3)--handle produced by Algorithm \ref{alg:build-3-handle} applied to $E'$\;
		}
		$W'' = $ the 4--manifold triangulation obtained by attaching $H_{E'}^3$ to $W'$ over $E'$ for each $E'\in\mathfrak{E}'$\;
		$M_1'' = \pd W''\setminus M_0$\;
		Apportion the cells of $\{H_{E'}^3\}_{E'\in\mathfrak{E}'}$ to the primed combinatorial vertex blocks of $W''$ to form double primed combinatorial vertex blocks\;
		$\mathfrak{V}'' = $ the collection of double primed combinatorial vertex blocks of $M_1''\subset\pd W''$\;
		\ForEach{$V''\in\mathfrak{V}''$}{
			$H_{V''}^4=$ the (4,4)--handle produced by Algorithm \ref{alg:build-4-handle} applied to $V''$\;
		}
		$W''' = $ the 4--manifold triangulation obtained by attaching $H_{V''}^4$ to $W''$ over $V''$ for each $V''\in\mathfrak{V}''$\;
		$\pd W'''=M_0$ and $M_0\equiv N$ through subdivision\;
		\textbf{return} $W'''$\;	
	}
\end{algorithm}



%\section{Attach 2--handles}
\label{section:pl-2-handle}

At this point in our procedure we have a 4--manifold $W$ with triangulated boundary components $M_0$ and $M_1$.
We aim to attach handles to $W$ over the boundary of $W$ away from $M_0$ until the boundary of $W$ is exactly $M_0$.
The first step is to attach 2--handles to $W$ over the closed solid torus triangulations that partition $M_1$.

The algorithm of this section takes as input a closed solid torus triangulation $T$ with a collection of triangulated parallel longitudes $\Gamma$ in its boundary (each $\gamma_i$ in $\Gamma$ is an explicit 0-framing for the 2--handle attachment) and produces as output a $D^4$ triangulation whose $S^3$ boundary triangulation has genus 1 Heegard splitting over $\pd T$, and $\gamma_i$ bounds a triangulated disk in $\pd D^4\setminus T$ for each $i$.
Such a $D^4$ triangulation is taken as a 4--dimensional 2--handle and attached to $W$ over $T$.
We then attach such a 2--handle over each closed solid torus in our partition of $M_0$.
%
%\section{Attach 3--handles}

We now have a triangulated 4--manifold with boundary components $M_0$ and the result of surgery on $M_1$ induced by 2--handle attachment.
This surgery had the following effect on $M_1$:
\begin{enumerate}
	\item Each solid torus in the partition of $M_1$ induced by the projection from Section ~\ref{section:pl-projection} is associated with a triangulated $D^4$ in Section ~\ref{section:pl-2-handle}.
	\item For each (torus, 2--handle) pair $(T, H^2)$, the boundary of $H^2$ contains $T$ and $T^* = H^2\setminus \interior{T}$ is another solid torus whose boundary triangulation is identical to that of $T$.
	\item Thus the effect of surgery on $M_1$ is of replacing each $T$ with $T^*$ over their shared boundary.
\end{enumerate}



%\section{Cone away final boundary components}


